\documentclass{article}
\usepackage{indentfirst}
\usepackage{lmodern}
\usepackage[utf8]{inputenc}
\usepackage[T1]{fontenc}
\usepackage[ngerman]{babel}
\usepackage{amssymb,amstext,amsmath}
\usepackage{graphicx}
\usepackage{dsfont}
\usepackage{amsfonts}
\usepackage{graphics}
\usepackage{float}
\usepackage{cite}
\usepackage{url}
\usepackage{tabularx}
\usepackage{capt-of}
 
\title{Fabry-Perot-Etalon}
\author{Alexander Heinisch, Dominik Wille}
\begin{document}
\maketitle

\begin{center}
\begin{minipage}{\linewidth}
\centering
\makebox[0cm]{\includegraphics[width=8cm]{bilder/FAP3}}
\label{ringe}
\end{minipage}
\end{center}
\vspace{7cm}
\noindent
\begin{center}
\begin{tabular}{r l}
Tutor & Diana Prychynenko  \\
Durchführung & 08. Mai 2013 von 14-18 Uhr\\

E-Mail Dominik & dominik.wille@fu-berlin.de \\
E-Mail Alexander & Matthias.Heinisch@gmx.de \\
\end{tabular}
\end{center}

\newpage
\tableofcontents
\newpage

\section{Ziele des Versuchs}
Experimentelle Einführung in das {\sc Fabry-Perot-Interferometer} als wichtiges Bauteil hochauflößender Spektralapparate und der Lasertechnik \\

\section{Physikalische Grundlagen}

\subsection{Vielstrahleninterferenz}
Bei der Beugung von Licht an einem Einfach- oder einem Doppelspalt entseht ein Interferenzmuster mit breiten Hauptmaxima und relativ dünnen Minima. Diese werden durch den Gangunterschied der Elementarwellen und die darauf folgende konstruktive bzw. destruktive Interferenz, hervorgerufen. Im Gegensatz dazu steht die Beugung des Lichts an einem optischen Gitter. Aufgrund der periodischen Struktur entstehen Vielfachinterferenzen mit sehr schmalen Interferenzmaxima und breiten Minima. Dies ermöglicht bei verschiedenen messtechnischen und spektroskopischen Anwendungen eine hohe Auflösung. 

\subsection{Fabry-Perot-Etalon}
Ein Fabry-Perot-Etalon (oder auch Fabry-Perlot-Interferrometer) ist ein optischer Resonator. Er besteht aus zwei parallelen, teilverspiegelten Flächen \(\ (G_{1} \) und \(\ G_{2} \)), zwischen denen eine einfallende ebene Welle \(\ (E_{e} \)) eine ''Zick-Zack-Reflexion'' erfährt. Da zwischen den Grenazflächen ein optisches Medium eingesperrt ist (z.b. Luft), wird der Lichtstrahl in kohärente Teilwellen \(\ (E_{i} \)) aufgespalten, die miteinander interferieren und eine auslaufende Welle \(\ (E_{a} \)) bildet. Den Gangunterschied berechnen wir mit

\begin{equation}
\label{1}
\delta = \overline{AC} + \overline{CD} - \overline{AB}= \frac{2d}{cos \alpha}- \frac {2d \cdot sin^2 \alpha}{cos \alpha} =2d\cdot cos \alpha
\end{equation}\\

Wobei A, B und C nacheinander die Punkte der Reflexion an den Spiegeln sind, \(\alpha \) ist der Einfallswinkel und d der Plattenabstand. Daraus folgt, dass sich der Gangunterschied verkleinert, wenn \(\alpha \) größer wird.

Des Weiteren können wir einen Gangunterschied durch Phasensprünge bei den Reflexionen vernachlässigen, da er bei Transmissionen einem ganzzahligen Vielfachen der Wellenlänge entspricht. Daraus folgt für transmittiertes Licht

\begin{equation}
\label{2}
\delta = 2 \cdot d \cdot cos \alpha = z \cdot \lambda \ \ \ mit\ z=1,2,3...
\end{equation}\\

Somit ist ersichtlich, dass nur dann Interfereinsmaxima entstehen, wenn \(\alpha \) oder \(\lambda \) Gl. \eqref{2} erfüllen.

\begin{center}
\begin{minipage}{\linewidth}
\centering
\makebox[0cm]{\includegraphics[width=8cm]{bilder/FAP1}}
\captionof{figure}{Aufbau Fabry-Perlot-Interferometer [GP 2 Skript]}%
\label{interferometer}
\end{minipage}
\end{center}

Die Phasengröße \(\phi \) bezeichnet den Gangunterschied in Einheiten der Wellenlänge, dich sich wie Folgt berechnet

\begin{equation}
\label{3}
\phi = \frac {\delta}{\lambda}
\end{equation}\\

Man nennt die ganzzahligen Werte Z von \(\phi \) Interferenzordnung der Maxima.

\subsection{Freier Spektralbereich}
In einem Dispersionsgebiet, kann man die ungestörten Wellenlängen eines Spektralapparats untersuchen. Als Bedingung für Interferenz gilt immer noch \eqref{2}, wobei man hier zwischen zwei Fälle unterscheidet. Entweder unterscheiden sich zwei benachbarte Maxima durch einen Ordnungsunterschied \(\Delta z =1 \) bei gleicher Wellenlänge, oder durch eine Wellenlängendifferenz \(\Delta \lambda \) bei gleicher Ordnung.
Somit wird aus \eqref{2} 

\begin{equation}
\label{4}
(z+1) \lambda=z( \lambda + \Delta \lambda)
\end{equation}
\vspace{0,25cm}

Daraus folgt für den freuen Spektralbereich des Etalons

\begin{equation}
\label{5}
\Delta \lambda = \frac {\lambda}{z} \  \ oder \ \ \frac {\Delta \lambda}{\lambda} = \frac {1}{z}
\end{equation}\\

Will man also ein nahezu monochromatisches Licht untersuchen oder eine Feinuntersuchung enger Wellenlängenbereiche durchführen, verwendet man den {\sc Fabry-Perot-Etalon}, welcher wegen seinem kleinen Dispersionsgebiet bei großen z besonders gut geeignet ist.

\subsection{Fabry-Perot-Spektrometer}
Wollen wir nun den Etalon als Spektrometer verwenden, müssen wir ihn mit divergentem Licht beleuchten, von dem unter einem bestimmten Einfallswinkel nur genau eine Farbe passieren kann. Dabei werden durch die Rotationssymmetrie der optischen Anordnung in der Brennebene der Linse konzentrische Ringe gleicher Neigung (Haudingersche Ringe) abgebildet. Nun setzten wir für den Neigungswinkel \(\alpha\) und \(cos\ \alpha \) näherungsweise

\begin{equation}
\label{6}
\alpha = \frac {r}{f} \ \ und \ \ cos \alpha = 1- \frac {1}{2} \alpha ^2
\end{equation}\\

für die Brennweite f der Linse und Radius r des Ringes, setzten wir Gl.\eqref{6} in Gl.\eqref{2} ein, woraus folgt:

\begin{equation}
\label{7}
z \approx \frac {2d}{\lambda} \left [ 1-\frac {r^2}{2f^2} \right ]
\end{equation}
oder:
\begin{equation}
\frac {2d}{\lambda} \approx z \left [1+ \frac {r^2}{2f^2} \right ]
\end{equation}\\

Um nun z aus den Gleichungen zu eliminieren, misst man bei bekannten Größen \(\ f, \lambda \) und \(\ d \) mindestens zwei Radien des Ringsystems und erhält für den Grenzflächenabstand

\begin{equation}
\label{8}
d = i \frac {\lambda f^2}{r_{i}^2 - r_{0}^2}
\end{equation}

mit i = 1,2,3,... dem Index des Ringes.
Sind\(d \) und\(z \) nicht bekannt, kann man die Wellenlänge \(\lambda \) und\(\lambda ' =\lambda +\Delta \lambda \) einer Ordnung mit den Radien \(r \) und\(r' \) berechnen. Wir benutzen Gl.\eqref{7} nach $\lambda$ umgestellt und subtrahieren $\lambda '$:

\begin{equation}
\lambda ' -\lambda =\frac{2d}{z} \cdot \frac{1}{2f^2}(r^2-r'^2)
\end{equation}
wobei $\frac{2d}{z}\approx \lambda$ gilt, ergibt das:
\begin{equation}
\label{9}
\Delta \lambda \approx \frac {\lambda}{2f^2} (r^2 - r'^2)
\end{equation}\\

berechnen. Die Herleitung erfolgt später in Aufgabe 2. Da die Ordnung eliminiert wurde, lässt die Genauigkeit nach, was zur Folge hat, dass man die Größen\(\ d \) und\(\ z \) möglichst genau bestimmen muss.\\

\subsection{Auflösungsvermögen des Fabry-Perot-Etalon} 
Will man die genaue Intensitätsverteilung um ein Interferenzmaximum in Abhängigkeit der Phasengröße \(\phi \) bestimmen, benötigt man die {\sc Airy-Formel}

\begin{equation}
\label{10}
\frac {I}{I_{0}} = \left [ \frac {T}{1-R} \right ]^2 \cdot \frac {1}{1+ \frac {4R}{(1-R)^2} \cdot sin^2 \phi \pi}
\end{equation}

mit dem Transmissionsgrad T und Reflexionsgrad R der Platten. Dafür folgt für die Halbwertsbreite \(\ 2 \Delta z \) der Interferenzmaxima näherungsweise für kleine Winkel

\begin{equation}
\label{11}
2 \Delta z= \frac {1-R}{\pi \sqrt {R}}
\end{equation}\\

\section{Aufgabenstellung}
\subsection*{Aufgabe 1}
Aufbau und Justierung der Apparatur.
\subsection*{Aufgabe 2}
Bestimmung des Plattenabstandes eines {\sc  Fabry-Perot-Etalon} mit der roten 643,9-nm-Linie von Cadmium und Berechnung der (ungefähren) Interferenzordnung.
\subsection*{Aufgabe 3}
Relative Bestimmung der Wellenlängen der grünen und der dunkelblauen Linie des Cadmium-Spektrums.
\subsection*{Aufgabe 4}
Abschätzung der Linienbreite der Interferenzmaxima für die rote Linie und Vergleich mit der erwarteten instrumentellen Linienbreite.

\newpage
\section{Aufgabe 1}
Bei dieser Aufgabe werden vor allem Einstellungen gesucht, bei denen eine Aufnahme der Franck-Hertz-Kurve in möglichst schöner Form möglich sind. Dabei hängt das Aussehen der kurve von verschiedenen Faktoren ab. Verwendet werden ein Funktionsgeneartor, der 
\begin{itemize}
\item[] \textbf{Heizspannung:} Da die Heizspannung für ein Austreten von Elektronen aus der Kathode sorgt, gibt es einen Zusammenhang zwischen Dem Strom von Anode zu Gegenkathode \(I\) und der Heizspannung \(U_H\). Es ist eine genügend große Heizspannung zu wählen um ein Austreten von Elektronen zu ermöglichen, ist diese erreicht hat eine Weitere Erhöhung nur noch wenig Effekt sorgt im Allgemeinen aber für eine Erhöhung des Stroms \(I\)
\end{itemize}
\newpage
\section{Aufgabe 2}

Aufgrund veränderter Umstände, haben wir die Frank-Hertz-Kurven nicht mehr mit einem X-Y-Schreiber aufgezeichnet, sondern mit einem Programm auf den Rechnern im Versuchsraum.

\subsection{Aufgabe 3}
\begin{figure}[H]
\includegraphics[scale=0.5]{sb3}
\caption{Abbildung 1: Schaltplan Aufgabe 3}
\end{figure}
In der letzten Aufgabe wurde eine Wechselstrombrücke aufgebaut um nun die Induktivität L einer Spule zu bestimmen. Der Aufbau besteht aus einem Phasenabgleichswiderstand R' mit einer bekannten und einer unbekannten Spule \(L_{0}\) und \(L_{x}\) in Reihe geschalten und dazu parallel ist die Wechselstrombrücke. Das Messgerät wird zwischen R' und der Wechselstrombrücke platziert. Das Ziel ist nun eigentlich, den Strom durch die Wechselstrombrücke auf Null zu stellen, was allerdings wegen den hier verwendeten Bauteilen nicht ganz möglich ist. Also versucht man ihn zu minimieren. Gemessen hatten wir mit dem Messgerät ELC-131D (\(\pm 2\% +5\) dgt) für \(\Delta L_{VS}\) und (\(\pm 0,5\% +3\) dgt) für \(\Delta R_{VS}\)\\
Bei einer Frequenz von f=1.9989 kHz erhielten wir für die gesuchten Größen folgende Werte:
\\

\begin{tabular}{l l}
\(R_{a}\) & =\(761,0\pm 4,3 \Omega\)\\
\(R_{b}\) & =\(241,8\pm 1,7 \Omega\)\\
\(R'\) & =\(3,700\pm 0,023 \Omega\)\\
\end{tabular}
\\

Für die Vergleichsspule gelten die Werte:\\

\begin{tabular}{l l}
\(R_{VS}\) =\((2,91\pm 0,02)\Omega\)\\
\(L_{VS}\) =\((1,507\pm 0,015)\)mH\\
\end{tabular}
\\

Nun überprüftt man mit dem Oszilloskop, ob die Spulen mit dem Funktionsgenerator in Phase sind.\\
\newpage
Des weiteren gilt es nun die Induktivität L und den Verlustwiderstand \(R_{V}\) der unbekannten Spule berechnen. Dazu nimmt man Gl.\(\eqref{L}\) für \(L_{x}\):
\begin{equation}\notag
L_x = {\frac{R_a}{R_b}} L_0=(4,75\pm 0,07)mH
\end{equation}

und auf Grund der Phasengleichheit errechnet sich für \(R_{V}\):
\begin{equation}\notag
R_{V}=\frac{R_{a}}{R_{b}}\cdot R_{VS}=(9,16\pm 0,11)\Omega
\end{equation}
Für die Fehler gilt:
\begin{equation}\notag
\Delta L_x = {\sqrt{\left({\frac{\partial L_x}{\partial R_a}}\Delta R_a\right)^2+\left({\frac{\partial L_x}{\partial R_b}}\Delta R_b\right)^2+\left({\frac{\partial L_x}{\partial L_0}}\Delta L_0\right)^2}}
\end{equation}
\begin{equation}\notag
\Delta R_{L_x} = {\sqrt{\left({\frac{\partial R_{L_x}}{\partial R_a}}\Delta R_a\right)^2+\left({\frac{\partial R_{L_x}}{\partial R_b}}\Delta R_b\right)^2+\left({\frac{\partial R_{L_x}}{\partial R_{L_0}}}\Delta R_{L_0}\right)^2}}
\end{equation}
\\

Als letztes wird noch der theoretische Werte für den Phasenunterschied bestimmt. Gemessen hatten wir für die Induktivität L:
\begin{equation}\notag
L=(4,823 \pm 0,038)mH
\end{equation}
\begin{equation}
\notag
R_{L}=(162,5 \pm 1,3)\Omega
\end{equation}

Dies nun in Gl.\(\eqref{d}\) eingesetzt mit \(\omega=2\pi f\) (f=199,98 Hz) und nach \(\phi\) aufgelöst, ergibt:
\begin{equation}
\notag
\phi=arctan \left(\frac{2\pi f L}{R_{L}}\right)=(2,14 \pm 0,03)^\circ
\end{equation}

Für den Fehler gilt wieder:
\begin{equation}\notag
\Delta \phi = {\sqrt{({\frac{\partial \phi}{\partial R_{L_1}}}\Delta R_{L_1})^2+({\frac{\partial \phi}{\partial L_1}}\Delta L_1)^2}}
\end{equation}

\subsection*{Fazit}
An sich verlief bei dieser Aufgabe alles soweit ganz gut, bis auf das kleine Zeitproblem was alle Gruppen bei diesem Versuch hatten. Da wir auch die einzige Gruppe waren, die den Aufbau zu dieser Aufgabe geschafft hatten, durften wir sozusagen den Versuch vorführen. Das ging dann allerdings ein wenig chaotisch zu, weil immer nachgefragt wurde was jetzt nochmal welcher Wert genau gewesen war. Nichts desto trotz sehen unsere Ergebnisse für Diesen Versuch ganz ordentlich aus. Das einzige was komisch ist, ist dass unser Widerstand der Vergleichsspule so klein ist. Da uns aber Vergleichswerte fehlen, beruht diese Aussage eher auf ein Bauchgefühl, als auf Tatsachen.\\
In der Berechnung des Fehlers der Phasenverschiebung haben wir den Fehler für die Frequenz weggelassen, weil er im Verhältnis zu denen der anderen Werte so verschwindend gering ist, dass es nicht ins Gewicht gefallen wäre.\\
Alles in allem verlief dieser Versuch zufriedenstellend.

\subsection{Aufgabe 4}
\subsubsection{Auswertung}
Nun sollen wir die Linienbreite schätzen, wozu wir den kleinsten erkennbaren Ring benutzten 
\\

\begin{tabular}{l|l|l|l|l|}
Ring i & \multicolumn{2}{c|}{\textbf{Ringabmessung rechts [mm]}} & \multicolumn{2}{c|}{\textbf{Ringabmessung links [mm]}} \\
		& Position innen & Position außen & Position innen & Position außen\\	
\hline
1 & $9.48\pm 0.17$ & $9.15\pm 0.17$ & $12.01\pm 0.19$ & $12.39\pm 0.19$ \\
& & & & \\
& Radius innen [mm] & Radius außen [mm] & Radius gemittelt [mm] &\\ \hline
1 & $1.27\pm 0.13$ & $1.62\pm 0.13$ & $1.45\pm 0.19$\\
\end{tabular}\\
\textbf{Tabelle 4.1:innerstes vermessene Maxima}\\
\\
Die Linienbreite L bestimmen wir nun über die Gl.$\eqref{9}$

\begin{equation}
\label{9}
L\approx \Delta \lambda \approx \frac {\lambda}{2f^2} (r^2 - r'^2)
\end{equation}\\
Der Fehler errechnet sich aus der Gauß'schen Fehlerfortpflanzung:

\begin{equation}
\Delta L=\sqrt{\frac{(r_{a}^2-r_{i}^2)^2\Delta \lambda^2}{4f^4}+\frac{(r_{a}^2-r_{i}^2)^2\Delta L^2\lambda^2}{f^6}+\frac{r_{a}^2\Delta r_{a}^2\lambda^2}{f^4}+\frac{r_{i}^2\Delta r_{i}^2\lambda^2}{f^4}}
\end{equation}
Daraus folgt für $L=(33.57\pm 2)pm$\\
\\
Um nun den theoretischen Wert zu berechnen, benutzen wir die Airy-Formel für näherungsweise kleine Winkel Gl.$\eqref{11}$

\begin{equation}\notag
L_{theo}=\frac{2 \Delta z\cdot \lambda}{z}=\frac{\frac {1-R}{\pi \sqrt {R}}\lambda}{z} 
\end{equation}\\
\\
Und für den Fehler:

\begin{equation}\notag
\begin{split}
\Delta L_{theo}=\sqrt{\left(\frac{\partial L_{theo}}{\partial R}\cdot \Delta R\right)^2+\left(\frac{\partial L_{theo}}{\partial z}\cdot \Delta z\right)^2}\\
\Delta L_{theo}=\sqrt{\frac{(1-R)^2\cdot\Delta z^2\lambda^2}{\pi^2\cdot R\cdot z^4}+\Delta R\left(\frac{-(1-R)\lambda}{2\pi R^{\frac{3}{2}}\cdot z}-\frac{\lambda}{\pi \cdot \sqrt{R}\cdot z}\right)^2}
\end{split}
\end{equation}

Mit $R=80\%$, $z=11374.2$ und $\Delta R=5$ (was wir nicht genau kannten und so festlegten) folgt für $L=(4.1\pm 1.3)pm$

\subsubsection{Fazit}
Die große Abweichung zwischen dem theoretischen Wert L=(4.1$\pm$1.3)pm und dem berechneten Wert L=(33$\pm$2)pm liegt wahrscheinlich größtenteils an der subjektiven Wahrnehmung des menschlichen Auges, da es, wie bei vielen optischen Versuchen, mit der Zeit immer anstrengender wird und die Konzentration nachlässt. Weitere Fehlerquellen sind das Auflösungsvermögen des Versuchsaufbaus,sowie die quantenmechanischen Effekte der Dopplerverbreiterung (Effekte, welche durch die thermische Bewegung der Atome auftreten) und die Druckverbreiterung (welche durch die verringerte Lebensdauer der Cadmium-Atome durch Stöße verursacht wird).
Noch weitere Ringe zu messen hätte die Auswertung genauer gemacht, aber schon die zehnte Ordnung war recht schwer zu erkennen.

\section{Fazit}
Wir sollten als erstes den Plattenabstand des Etalons berechnen und kamen dort auf einen Wert von \(d=(3,7\pm 0,2)mm\). Wir konnten ihn mangels Vergleichswerte nicht überprüfen aber er scheint uns trotzdem als realistisch.\\

In Aufgabe 3 bestimmten wir die Wellenlänge der grünen und dunkelblauen Linie des Cadmium-Spektrums. Im Allgemeinen verlief die Messung selbst relativ gut, nur haben wir entweder dort oder eben bei der Auswertung der Messdaten einen gravierenden Fehler begangen, sodass wir auf vollkommen verkehrte Werte für die roten bzw. blauen Wellenlängen kamen. Ein Versuch das zu korrigieren war\\

Die letzte Aufgabe ergab für die Linienbreite $L=(33.57\pm 2)pm$ und für die Größe $z=11359,1$, was an sich realistisch und somit gute Ergebnisse sind.\\

Aufgabe 1 und 4 ergaben also befriedigende Ergebnisse. Auch bei Aufgabe 2 sieht das Ergebnis gut aus, wobei uns ja hier, wie schon zuvor erwähnt, Vergleichswerte fehlen, um dies mit Bestimmtheit sagen zu können.

\section{Qellenangabe}
\begin{itemize}
\item GPII-Skript
\item http://physics.nist.gov/PhysRefData/Handbook/Tables/cadmiumtable2.htm
\end{itemize}

\vspace{7.0cm}

\begin{tabularx}{\textwidth}[b]{p{5cm} X p{5cm}} \cline{1-1} \cline{3-3}
Datum, Dominik Wille & & Datum, Alexander Heinisch
\end{tabularx}
\end{document}