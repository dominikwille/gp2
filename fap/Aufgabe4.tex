\subsection{Auswertung}
Nun sollen wir die Linienbreite schätzen, wozu wir den kleinsten erkennbaren Ring benutzten 
\\

\begin{tabular}{l|l|l|l|l}
 & \multicolumn{2}{c|}{\textbf{Ringabmessung rechts [mm]}} & \multicolumn{2}{c}{\textbf{Ringabmessung links [mm]}} \\
Ring i & & & & \\
		& Position innen & Position außen & Position innen & Position außen\\	
\hline
 & $9.48\pm 0.17$ & $9.15\pm 0.17$ & $12.01\pm 0.19$ & $12.39\pm 0.19$ \\
1 & & & & \\
& Radius innen [mm] & Radius außen [mm] & Radius gemittelt [mm] &\\ \hline
& & &  \\
1 & $1.27\pm 0.13$ & $1.62\pm 0.13$ & $1.45\pm 0.19$\\
\end{tabular}\\
\begin{center}
\it Tab. 4.1: innerstes vermessene Maxima\\
\end{center}

Die Linienbreite L bestimmen wir nun über die Gl.$\eqref{9}$

\begin{equation}\notag
L\approx \Delta \lambda \approx \frac {\lambda}{2f^2} (r^2 - r'^2)
\end{equation}\\
Der Fehler errechnet sich aus der Gauß'schen Fehlerfortpflanzung:

\begin{equation}
\Delta L=\sqrt{\frac{(r_{a}^2-r_{i}^2)^2\Delta \lambda^2}{4f^4}+\frac{(r_{a}^2-r_{i}^2)^2\Delta L^2\lambda^2}{f^6}+\frac{r_{a}^2\Delta r_{a}^2\lambda^2}{f^4}+\frac{r_{i}^2\Delta r_{i}^2\lambda^2}{f^4}}
\end{equation}
Daraus folgt für $L=(33.57\pm 2)pm$\\
\\
Um nun den theoretischen Wert zu berechnen, benutzen wir die Airy-Formel für näherungsweise kleine Winkel Gl.$\eqref{11}$

\begin{equation}\notag
L_{theo}=\frac{2 \Delta z\cdot \lambda}{z}=\frac{\frac {1-R}{\pi \sqrt {R}}\lambda}{z} 
\end{equation}\\
\\
Und für den Fehler:

\begin{equation}\notag
\begin{split}
\Delta L_{theo}=\sqrt{\left(\frac{\partial L_{theo}}{\partial R}\cdot \Delta R\right)^2+\left(\frac{\partial L_{theo}}{\partial z}\cdot \Delta z\right)^2}\\
\Delta L_{theo}=\sqrt{\frac{(1-R)^2\cdot\Delta z^2\lambda^2}{\pi^2\cdot R\cdot z^4}+\Delta R\left(\frac{-(1-R)\lambda}{2\pi R^{\frac{3}{2}}\cdot z}-\frac{\lambda}{\pi \cdot \sqrt{R}\cdot z}\right)^2}
\end{split}
\end{equation}

Mit $R=80\%$, $z=11374.2$ und $\Delta R=5$ (was wir nicht genau kannten und so festlegten) folgt für $L=(4.1\pm 1.3)pm$

\subsection{Fazit}
Die große Abweichung zwischen dem theoretischen Wert L=(4.1$\pm$1.3)pm und dem berechneten Wert L=(33$\pm$2)pm liegt wahrscheinlich größtenteils an der subjektiven Wahrnehmung des menschlichen Auges, da es, wie bei vielen optischen Versuchen, mit der Zeit immer anstrengender wird und die Konzentration nachlässt. Weitere Fehlerquellen sind das Auflösungsvermögen des Versuchsaufbaus,sowie die quantenmechanischen Effekte der Dopplerverbreiterung (Effekte, welche durch die thermische Bewegung der Atome auftreten) und die Druckverbreiterung (welche durch die verringerte Lebensdauer der Cadmium-Atome durch Stöße verursacht wird).
Noch weitere Ringe zu messen hätte die Auswertung genauer gemacht, aber schon die zehnte Ordnung war recht schwer zu erkennen.