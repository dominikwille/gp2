\section{Aufgabe 2.1}
\subsection{Abschätzung des Arbeitswiderstandes}

Um den Arbeitswiderstand zu bestimmen, benutzen wir das Vier-Quadranten-Kennlinienfeld (Anhang Abbildung 2.1) und gehen vom äußersten Punkt des dritten Quadranten für \(I_B\)= 140,8\(\mu A\) senkrecht nach oben. Im Zweiten Quadranten treffen wir so den Wert für \(I_C\)=13,9 mA, welches unser Startpunkt der Arbeitsgeraden (bzw. Kollektor-Widerstandsgerade) sein wird. Damit ergeben sich die Koordinaten (\(U_{CE}\) = 0[V], \(I_{C}\) = 13,9[mA]) für den Punkt \(P_1\) des Kurzschlussfalls. Der Punkt \(P_2\) liegt bei den Koordinaten (\(I_{C}\) = 0 [mA],\\
\(U_{CE}\) = 12[V]), weil 12[V] die maximale Versorgungsspannung betrug.\\
Der Arbeitspunkt \(P_{A}\) liegt laut Definition bei \(\left(\frac{U_{CE}}{2}, \frac{I_{C}}{2}\right)\), welches bei uns den Werten (6.055[V],11.7[mA]) entspricht.\\

Den Arbeitswiderstand wird über die Formel:
\begin{equation}
\notag
R_{A}=\mid \frac{1}{m}\mid = \left(\frac{U_{EC}}{I_C}\right) = 517.54 \Omega
\end{equation}\\
Auch wenn es sich um einen abgeschätzten Wert handelt, berechnen wir den Fehler über die Gauß'sche Fehlerfortpflanzung:
\begin{equation}
\notag
\Delta R_{A}=\sqrt{\left(\frac{\Delta I_C}{I_C}\right)^2+\left(\frac{\Delta U_{EC}}{U_{EC}}\right)^2}\cdot R_{A} = 4.53\Omega
\end{equation}

Während des Versuchs haben wir für den Arbeitswiderstand einen Wert von (\(464.3\pm 2.6)\Omega\) gemessen.\\

Der Basisvorwiderstand ergibt sich aus der Formel:
\begin{equation}
\notag
R_V= \frac{U_{EC}-U_{EB}}{I_B}=\frac{6V-0.59V}{(113.4\cdot 10^{-6})A}=47.71k\Omega
\end{equation}