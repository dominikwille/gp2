\subsection{Aufgabe 3}
\subsubsection{Aufbau}
Die Apparatur wurde aus Aufgabe 1 übernommen. Die Spaltbreite wurde auf \(0,2\,mm\) eingestellt und anschließend wurde der Messung begonnen.
\subsubsection{Durchführung}
Vermessen wurde als erstes die Indigo-Linie des Farbspektrums, da diese schon aus der vorherigen Aufgabe eingestellt war, anschließend folgten die grüne- und gelbe Linie. Die Gegenspannung \(U_G\) wurde langsam abgebaut und die gemessenen Ströme \(I_{Pho}\) notiert. Die Messabstände wurden so gewählt, dass bei relativ konstantem Verlauf nur wenig Messwerte genommen wurden, bei größeren Änderungen aber viele.
\subsubsection{Beobachtung}
Bis zu einer Gewissen Spannung trat kaum eine Änderung des Stroms \(I_{Pho}\) auf, anschließend jedoch sehr schnell. Bis zu einer gewssen Spannu ng stieg der Strom immer weiter blieb dann jedoch realtiv konstant.
\subsubsection{Messwerte}
\begin{center}
\begin{tabular}{c c}
\(-8\) & \( -0.5\) \\ 
\(-7\) & \( -0.6\) \\ 
\(-6\) & \( -0.6\) \\ 
\(-5\) & \( -0.6\) \\ 
\(-4\) & \( -0.6\) \\ 
\(-3\) & \( -0.6\) \\ 
\(-2\) & \( -0.6\) \\ 
\(-1.5\) & \( -0.6\) \\ 
\(-1.1\) & \( -0.5\) \\ 
\(-1\) & \( -0.5\) \\ 
\(-0.9\) & \( -0.26\) \\ 
\(-0.8\) & \( -0.06\) \\ 
\(-0.7\) & \( 0.47\) \\ 
\(-0.6\) & \( 1.08\) \\ 
\(-0.5\) & \( 1.8\) \\ 
\(-0.4\) & \( 2.7\) \\ 
\(-0.3\) & \( 3.9\) \\ 
\(-0.2\) & \( 5.3\) \\ 
\(-0.1\) & \( 7\) \\ 
\(0\) & \( 8.95\) \\ 
\(0.2\) & \( 14.2\) \\ 
\(0.4\) & \( 19.8\) \\ 
\(0.6\) & \( 26\) \\ 
\(0.8\) & \( 32.6\) \\ 
\(1\) & \( 33.6\) \\ 
\(1.2\) & \( 42.9\) \\ 
\(1.4\) & \( 52.2\) \\ 
\(1.6\) & \( 60.3\) \\ 
\(1.8\) & \( 69\) \\ 
\(2\) & \( 76.4\) \\ 
\(2.2\) & \( 84.2\) \\ 
\(2.4\) & \( 91.4\) \\ 
\(2.6\) & \( 98.4\) \\ 
\(2.8\) & \( 104\) \\ 
\(3\) & \( 112\) \\ 
\(3.2\) & \( 118\) \\ 
\(3.4\) & \( 123\) \\ 
\(3.6\) & \( 127\) \\ 
\(3.8\) & \( 130\) \\ 
\(4\) & \( 135\) \\ 
\(4.5\) & \( 142\) \\ 
\(5\) & \( 148\) \\ 
\(5.5\) & \( 153\) \\ 
\(6\) & \( 157\) \\ 
\(6.5\) & \( 162\) \\ 
\(7\) & \( 167\) \\ 
\(7.5\) & \( 170\) \\ 
\(8\) & \( 174\) \\
\end{tabular}
\captionof{table}{Messwerte der Indigo-Linie}
\begin{tabular}{c c}
\(-8\) & \( 0.2\) \\ 
\(-7.8\) & \( 0.1\) \\ 
\(-7.6\) & \( 0.1\) \\ 
\(-7.4\) & \( 0.05\) \\ 
\(-7.2\) & \( 0.06\) \\ 
\(-7\) & \( 0.04\) \\ 
\(-6.8\) & \( 0.06\) \\ 
\(-6.6\) & \( 0.05\) \\ 
\(-6\) & \( 0.05\) \\ 
\(-5\) & \( 0.03\) \\ 
\(-4\) & \( 0.04\) \\ 
\(-3\) & \( 0.1\) \\ 
\(-2\) & \( 0.16\) \\ 
\(-1\) & \( 0.2\) \\ 
\(-0.5\) & \( 0.25\) \\ 
\(0\) & \( 0.8\) \\ 
\(0.2\) & \( 1.7\) \\ 
\(0.4\) & \( 4.5\) \\ 
\(0.6\) & \( 10.3\) \\ 
\(0.8\) & \( 20.98\) \\ 
\(1\) & \( 39.4\) \\ 
\(1.2\) & \( 66.9\) \\ 
\(1.4\) & \( 106\) \\ 
\(1.6\) & \( 141\) \\ 
\(1.8\) & \( 175\) \\ 
\(2\) & \( 208\) \\ 
\(2.2\) & \( 242\) \\ 
\(2.4\) & \( 267\) \\ 
\(2.6\) & \( 286\) \\ 
\(2.8\) & \( 202\) \\ 
\(3\) & \( 316\) \\ 
\(3.2\) & \( 330\) \\ 
\(3.4\) & \( 341\) \\ 
\(3.6\) & \( 351\) \\ 
\(6.8\) & \( 359\) \\ 
\(4\) & \( 372\) \\ 
\(5\) & \( 391\) \\ 
\(6\) & \( 409\) \\ 
\(7.8\) & \( 419\) \\ 
\(8\) & \( 435\) \\
\end{tabular}
\captionof{table}{Messwerte der grünen Linie}
\end{center}
\subsubsection{Graphen}
