\subsection{Aufgabe 4}

In unserem Versuch der Photoemission, werden einige Widersprüche zur klassischen Wellentheorie deutlich und unsere erhaltenen Ergebnisse stimmen mit Einsteins Quantentheorie überein.\\

Die klassische Wellentheorie besagt zum Beispiel, dass es keinen Zusammenhang zwischen der Intensität des Photostroms und der Anzahl der herausgelösten Elektronen des Metalls gibt.\\

Auch besagt sie, dass die kinetische Energie der Photonen direkt von der Intensität des einfallenden Lichts abhängig ist. Wenn ein Elektron von einer oszillierenden Schwingung angeregt werden würde, müsste es das Metall verlassen, wenn eine kritische Amplitude überschritten wäre. Das Elektron hätte also demnach eine gewisse kinetische Energie.
\\
Die Erkenntnisse der Photoemission belegen jedoch, dass in Wirklichkeit die Frequenz des Lichts einen direkten Einfluss auf die kinetische Energie der Elektronen hat, und die Intensität des Lichts das Maß des Elektronenstroms bestimmt.