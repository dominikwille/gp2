\subsection{Aufgabe 4}

In unserem Versuch der Photoemission, werden einige Widersprüche zur klassischen Wellentheorie deutlich und unsere erhaltenen Ergebnisse stimmen mit Einsteins Quantenhypothese überein.\\

Je höher die Intensität des Lichts, desto höher ist die Anzahl emittierten Elektronen. Das konnte im ersten Teil von Aufgabe 2 gezeigt werden, bei dem durch Intensitätserhöhung mit einer Saugspannung eindeutig ein höherer Strom gemessen wurde.

Die Quantenhypothese besagt, dass es keinen Zusammenhang zwischen der Intensität des Photostroms und der Übertragenen Energie gibt. Das konnte eindrucksvoll dadurch bestätigt werden, dass trotz größerer Intensität des einfallenden Lichts, bei bestimmten Gegenspannungen kein Strom in Aufgabe 2 messbar war.


Die Wellentheorie besagt dagegen, dass die kinetische Energie der Photonen direkt von der Intensität des einfallenden Lichts abhängt, was nicht der Fall ist. Wenn ein Elektron von einer oszillierenden Schwingung angeregt werden würde, müsste es das Metall verlassen, wenn eine kritische Amplitude überschritten wäre. Das Elektron hätte also demnach eine gewisse kinetische Energie. Des weiteren hat diese Schwingung die Folge, dass es erst kurze Zeit nach dem Einschalten einen Messbaren Elektronenstrom gibt. Unseren Beobachtungen zufolge, war nach dem Einschalten instantan ein konstanter Elektronenstrom vorhanden.\\
Die Erkenntnisse der Photoemission belegen jedoch, dass in Wirklichkeit die Frequenz des Lichts einen direkten Einfluss auf die kinetische Energie der Elektronen hat.\\
Auch die in Aufgabe 3 erhaltenen Messwerte liefern einen eindeutigen Zusammenhang zwischen Frequenz und Bremsspannung, was der klassischen Ansicht nach unmöglich ist.