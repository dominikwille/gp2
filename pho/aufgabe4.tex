\subsection{Aufgabe 4}

In unserem Versuch der Photoemission, werden einige Widersprüche zur klassischen Wellentheorie deutlich und unsere erhaltenen Ergebnisse stimmen mit Einsteins Quantenhypothese überein.\\

Die klassische Wellentheorie besagt zum Beispiel, dass es keinen Zusammenhang zwischen der Intensität des Photostroms und der Anzahl der herausgelösten Elektronen des Metalls gibt.\\
Dies wurde durch die in Aufgabe 2 gemessenen Werte eindeutig widerlegt. Je höher die Intensität des Photostroms, desto höher ist auch das Maß der emittierten Elektronen.\\
Auch besagt die Wellentheorie, dass die kinetische Energie der Photonen direkt von der Intensität des einfallenden Lichts abhängig ist. Wenn ein Elektron von einer oszillierenden Schwingung angeregt werden würde, müsste es das Metall verlassen, wenn eine kritische Amplitude überschritten wäre. Das Elektron hätte also demnach eine gewisse kinetische Energie. Des weiteren hat diese Schwingung die Folge, dass es erst kurze Zeit nach dem Einschalten einen Messbaren Elektronenstrom gibt. Unseren Beobachtungen zufolge, war nach dem Einschalten instantan ein konstanter Elektronenstrom vorhanden.\\
Die Erkenntnisse der Photoemission belegen jedoch, dass in Wirklichkeit die Frequenz des Lichts einen direkten Einfluss auf die kinetische Energie der Elektronen hat.\\
Auch die in Aufgabe 3 erhaltenen Messwerte liefern einen eindeutigen Zusammenhang zwischen Frequenz und Bremsspannung, was der klassischen Ansicht nach unmöglich ist.