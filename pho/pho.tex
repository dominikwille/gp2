\documentclass{article}
\usepackage{indentfirst}
\usepackage{lmodern}
\usepackage[utf8]{inputenc}
\usepackage[T1]{fontenc}
\usepackage[ngerman]{babel}
\usepackage{amssymb,amstext,amsmath}
\usepackage{graphicx}
\usepackage{dsfont}
\usepackage{amsfonts}
\usepackage{graphics}
\usepackage{float}
\usepackage{cite}
\usepackage{url}
\usepackage{tabularx}
\usepackage{capt-of}

\title{Photoemission}
\author{Alexander Heinisch, Dominik Wille}
\begin{document}
\maketitle
\vspace{13cm}
\noindent
\begin{center}
\begin{tabular}{r l}
Tutor & Heimberger\\
Durchführung & 29. Mai 2013 von 14-18 Uhr \\

E-Mail Dominik & dominik.wille@fu-berlin.de \\
E-Mail Alexander & matthias.heinisch@gmx.de \\
\end{tabular}
\end{center}

\newpage
\tableofcontents
\newpage

\section{Physikalische Grundlagen}
\newpage
\section{Aufgaben}

\subsection{Aufgabe 1}
Aufbau und Justierung der Apparatur

\subsection{Aufgabe 2}
Messung des Sättigungsstromes und der Bremsspannung einer Kalium-Photozelle in Abhängigkeit von der Beleuchtungsstärke für die 436-nm-Linie Iindigo/blau) von Quecksilber.

\subsection{Aufgabe 3}
Aufnahme der Strom-Spannungs-Kennlinien für alle Hauptlinien des Quecksilber-Spektrums. Auswertung der Kennlinien und Bestimmung des Plankschen Wirkungsquantums und der Austrittsarbeit von Kalium.

\subsection{Aufgabe 4}
Theoretische Aufgabe für die Ausarbeitung: Darstellung der Widersprüche zwischen den experimentellen Ergebnissen der Photoemission und der klassischen Wellentheorie des Lichts.

\newpage
\end{document}