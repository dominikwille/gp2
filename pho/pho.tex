\documentclass{article}
\usepackage{indentfirst}
\usepackage{lmodern}
\usepackage[utf8]{inputenc}
\usepackage[T1]{fontenc}
\usepackage[ngerman]{babel}
\usepackage{amssymb,amstext,amsmath}
\usepackage{graphicx}
\usepackage{dsfont}
\usepackage{amsfonts}
\usepackage{graphics}
\usepackage{float}
\usepackage{cite}
\usepackage{url}
\usepackage{tabularx}
\usepackage{capt-of}

\title{Photoemission}
\author{Alexander Heinisch, Dominik Wille}
\begin{document}
\maketitle

\begin{center}
\begin{minipage}{\linewidth}
\centering
\makebox[0cm]{\includegraphics[width=5cm]{bilder/pho0}}
\label{wtd}
\end{minipage}
\end{center}

\vspace{7cm}
\noindent
\begin{center}
\begin{tabular}{r l}
Tutor & Heimberger\\
Durchführung & 29. Mai 2013 von 14-18 Uhr \\

E-Mail Dominik & dominik.wille@fu-berlin.de \\
E-Mail Alexander & Matthias.Heinisch@gmx.de \\
\end{tabular}
\end{center}

\newpage
\tableofcontents
\newpage

\section{Physikalische Grundlagen}

\subsection{Hallwachs-Effekt}
Ende des 19. Jahrhunderts entdeckte Wilhelm Hallwachs, dass die Bestrahlung einer Metallplatte mit kurzwelligem Licht eine Ladungstrennung zur Folge hat. Die Platte lädt sich dabei durch die Abgabe von negativen Ladungen, bis zu einer gewissen Spannung, positiv auf. Ist dieses Haltepotential erreicht, werden weitere ausgelöste Ladungen zurückgehalten (Hallwachs-Effekt). Später fand man heraus, dass die herausgelösten Ladungen Elektronen sind (Photoemission).
Diese Entdeckung stand im Widerspruch zur damals gängigen Wellentheorie des Lichts und durch intensive Untersuchungen gelangte man zu folgenden Ergebnissen:

\begin{itemize}
\item Der Effekt tritt instatan auf
\item Es gibt einen proportionalen Zusammenhang zwischen dem Sättigungsstrom und der Intensität des Lichts
\item Es gibt keinen Zusammenhang zwischen der Intensität des Lichts und der kinetischen Energie der herausgelösten Elektronen
\item Je höher die Frequenz des Lichts, desto höher ist die kinetische Energie der Elektronen
\item Die Frequenz muss einen Minimalwert überschreiten, damit der Effekt eintritt (langweilige Frequenz)
\end{itemize}

\subsection{Licht-Quantentheorie}
Aufgrund der Unstimmigkeiten der Wellentheorie des Lichts mit den neuen Entdeckungen, schlug Albert Einstein 1905 eine neue Korpuskulartheorie des Lichts vor.\\
Licht besteht aus Lichtquanten, welche sich geradlinig mit der Geschwindigkeit c bewegen. In Abhängigkeit der Frequenz, tragen die Quanten dabei eine Energie von \(h\cdot \nu\). Kommt es nun zur Photoemission, wird die gesamte Energie des Lichtteilchens in Form von kinetischer Energie und verrichteter Austrittsarbeit auf das Elektron übertragen. 

\subsection{Austrittsarbeit}
Um ein Elektron aus einem (metallenem) Festkörper zu lösen, muss ihm Energie zugeführt werden. Die dabei benötigte Menge an Energie nennt man Austrittsarbeit. Je nach Metall, benötigt man weniger Energie (etwa 2eV im Frequenzbereich des sichtbaren Lichts für Alkalimetalle) oder mehr Energie (4-5eV im ultravioletten Bereich), um die langweilige Grenze zu erreichen. 

\subsection{Experimentelle Anordnung}
Der einfachste experimentelle Aufbau einer Photozelle, besteht aus einer flächenförmigen Kathode und einer Gegenelektrode. Die Kathode wir aus dem zu untersuchenden Material bestehen und die Elektrode das Gegenpotential aufbauen und den Photostrom auffangen.

\begin{center}
\begin{minipage}{\linewidth}
\centering
\makebox[0cm]{\includegraphics[width=8cm]{bilder/pho1}}
\captionof{figure}{Prinzipschaltskizze einer Photozelle [GP2-Skript]}%
\label{photozelle}
\end{minipage}
\end{center}

In der Skizze werden Elektronen durch den von links kommenden Photonenstrom der Energie \(h\cdot \nu\) aus der Kathode gelöst. Um die freien Elektronen aufzufangen, wird an der Elektrode eine Spannung U angelegt, durch welche auch der Photonenstrom \(I_{Ph}\) gemessen werden kann. Meistens werden Alkalimetalle für diese Experimente benutzt, da der Quantenausbeute bei ihnen ziemlich hoch ist (Verhältnis zwischen Photonen und emittierten Elektronen). Die beobachtbare Ströme werden im Bereich von \(10^{-12}\) bis \(10^{-9}\) Ampere liegen.

\subsection{Gegenfeldmethode}
Um die kinetische Energie der Elektronen messen zu können, verwenden wir die sogenannte Gegenfeldmethode. Sobald Elektronen zur Anode gelangen, ist ein Strom messbar. Wird nun eine Gegenspannung zwischen Kathode und Anode angelegt, wird der Elektronenfluss gebremst bis der Photostrom zum Stillstand kommt (\(I_{Ph}\)=0). Zwischen Gegenspannung und kinetischer Energie besteht folgende Relation:
\begin{equation}
E_{kin}=\frac{m_e}{2}\nu^2 =e\cdot U_{geg} = E_{max}
\end{equation}

Trägt man nun \(E_{max}\) gegen die Frequenz \(\nu\) auf, ergibt sich als Steigung das Planksche Wirkungsquantum. Für \(E_{max}\)=0 erhält man die Grenzfrequenz des untersuchten Metalls.
\newpage
\section{Aufgaben}

\subsection{Aufgabe 1}
Aufbau und Justierung der Apparatur

\subsection{Aufgabe 2}
Messung des Sättigungsstromes und der Bremsspannung einer Kalium-Photozelle in Abhängigkeit von der Beleuchtungsstärke für die 436-nm-Linie (Iindigo/blau) von Quecksilber.

\subsection{Aufgabe 3}
Aufnahme der Strom-Spannungs-Kennlinien für alle Hauptlinien des Quecksilber-Spektrums. Auswertung der Kennlinien und Bestimmung des Plankschen Wirkungsquantums und der Austrittsarbeit von Kalium.

\subsection{Aufgabe 4}
Theoretische Aufgabe für die Ausarbeitung: Darstellung der Widersprüche zwischen den experimentellen Ergebnissen der Photoemission und der klassischen Wellentheorie des Lichts.

\newpage
\section{Auswertung}
\section{Aufgabe 1}
Bei dieser Aufgabe werden vor allem Einstellungen gesucht, bei denen eine Aufnahme der Franck-Hertz-Kurve in möglichst schöner Form möglich sind. Dabei hängt das Aussehen der kurve von verschiedenen Faktoren ab. Verwendet werden ein Funktionsgeneartor, der 
\begin{itemize}
\item[] \textbf{Heizspannung:} Da die Heizspannung für ein Austreten von Elektronen aus der Kathode sorgt, gibt es einen Zusammenhang zwischen Dem Strom von Anode zu Gegenkathode \(I\) und der Heizspannung \(U_H\). Es ist eine genügend große Heizspannung zu wählen um ein Austreten von Elektronen zu ermöglichen, ist diese erreicht hat eine Weitere Erhöhung nur noch wenig Effekt sorgt im Allgemeinen aber für eine Erhöhung des Stroms \(I\)
\end{itemize}

\section{Aufgabe 2}

Aufgrund veränderter Umstände, haben wir die Frank-Hertz-Kurven nicht mehr mit einem X-Y-Schreiber aufgezeichnet, sondern mit einem Programm auf den Rechnern im Versuchsraum.
\subsection{Aufgabe 3}
\begin{figure}[H]
\includegraphics[scale=0.5]{sb3}
\caption{Abbildung 1: Schaltplan Aufgabe 3}
\end{figure}
In der letzten Aufgabe wurde eine Wechselstrombrücke aufgebaut um nun die Induktivität L einer Spule zu bestimmen. Der Aufbau besteht aus einem Phasenabgleichswiderstand R' mit einer bekannten und einer unbekannten Spule \(L_{0}\) und \(L_{x}\) in Reihe geschalten und dazu parallel ist die Wechselstrombrücke. Das Messgerät wird zwischen R' und der Wechselstrombrücke platziert. Das Ziel ist nun eigentlich, den Strom durch die Wechselstrombrücke auf Null zu stellen, was allerdings wegen den hier verwendeten Bauteilen nicht ganz möglich ist. Also versucht man ihn zu minimieren. Gemessen hatten wir mit dem Messgerät ELC-131D (\(\pm 2\% +5\) dgt) für \(\Delta L_{VS}\) und (\(\pm 0,5\% +3\) dgt) für \(\Delta R_{VS}\)\\
Bei einer Frequenz von f=1.9989 kHz erhielten wir für die gesuchten Größen folgende Werte:
\\

\begin{tabular}{l l}
\(R_{a}\) & =\(761,0\pm 4,3 \Omega\)\\
\(R_{b}\) & =\(241,8\pm 1,7 \Omega\)\\
\(R'\) & =\(3,700\pm 0,023 \Omega\)\\
\end{tabular}
\\

Für die Vergleichsspule gelten die Werte:\\

\begin{tabular}{l l}
\(R_{VS}\) =\((2,91\pm 0,02)\Omega\)\\
\(L_{VS}\) =\((1,507\pm 0,015)\)mH\\
\end{tabular}
\\

Nun überprüftt man mit dem Oszilloskop, ob die Spulen mit dem Funktionsgenerator in Phase sind.\\
\newpage
Des weiteren gilt es nun die Induktivität L und den Verlustwiderstand \(R_{V}\) der unbekannten Spule berechnen. Dazu nimmt man Gl.\(\eqref{L}\) für \(L_{x}\):
\begin{equation}\notag
L_x = {\frac{R_a}{R_b}} L_0=(4,75\pm 0,07)mH
\end{equation}

und auf Grund der Phasengleichheit errechnet sich für \(R_{V}\):
\begin{equation}\notag
R_{V}=\frac{R_{a}}{R_{b}}\cdot R_{VS}=(9,16\pm 0,11)\Omega
\end{equation}
Für die Fehler gilt:
\begin{equation}\notag
\Delta L_x = {\sqrt{\left({\frac{\partial L_x}{\partial R_a}}\Delta R_a\right)^2+\left({\frac{\partial L_x}{\partial R_b}}\Delta R_b\right)^2+\left({\frac{\partial L_x}{\partial L_0}}\Delta L_0\right)^2}}
\end{equation}
\begin{equation}\notag
\Delta R_{L_x} = {\sqrt{\left({\frac{\partial R_{L_x}}{\partial R_a}}\Delta R_a\right)^2+\left({\frac{\partial R_{L_x}}{\partial R_b}}\Delta R_b\right)^2+\left({\frac{\partial R_{L_x}}{\partial R_{L_0}}}\Delta R_{L_0}\right)^2}}
\end{equation}
\\

Als letztes wird noch der theoretische Werte für den Phasenunterschied bestimmt. Gemessen hatten wir für die Induktivität L:
\begin{equation}\notag
L=(4,823 \pm 0,038)mH
\end{equation}
\begin{equation}
\notag
R_{L}=(162,5 \pm 1,3)\Omega
\end{equation}

Dies nun in Gl.\(\eqref{d}\) eingesetzt mit \(\omega=2\pi f\) (f=199,98 Hz) und nach \(\phi\) aufgelöst, ergibt:
\begin{equation}
\notag
\phi=arctan \left(\frac{2\pi f L}{R_{L}}\right)=(2,14 \pm 0,03)^\circ
\end{equation}

Für den Fehler gilt wieder:
\begin{equation}\notag
\Delta \phi = {\sqrt{({\frac{\partial \phi}{\partial R_{L_1}}}\Delta R_{L_1})^2+({\frac{\partial \phi}{\partial L_1}}\Delta L_1)^2}}
\end{equation}

\subsection*{Fazit}
An sich verlief bei dieser Aufgabe alles soweit ganz gut, bis auf das kleine Zeitproblem was alle Gruppen bei diesem Versuch hatten. Da wir auch die einzige Gruppe waren, die den Aufbau zu dieser Aufgabe geschafft hatten, durften wir sozusagen den Versuch vorführen. Das ging dann allerdings ein wenig chaotisch zu, weil immer nachgefragt wurde was jetzt nochmal welcher Wert genau gewesen war. Nichts desto trotz sehen unsere Ergebnisse für Diesen Versuch ganz ordentlich aus. Das einzige was komisch ist, ist dass unser Widerstand der Vergleichsspule so klein ist. Da uns aber Vergleichswerte fehlen, beruht diese Aussage eher auf ein Bauchgefühl, als auf Tatsachen.\\
In der Berechnung des Fehlers der Phasenverschiebung haben wir den Fehler für die Frequenz weggelassen, weil er im Verhältnis zu denen der anderen Werte so verschwindend gering ist, dass es nicht ins Gewicht gefallen wäre.\\
Alles in allem verlief dieser Versuch zufriedenstellend.
\subsection{Aufgabe 4}

In unserem Versuch der Photoemission, werden einige Widersprüche zur klassischen Wellentheorie deutlich und unsere erhaltenen Ergebnisse stimmen mit Einsteins Quantenhypothese überein.\\

Die klassische Wellentheorie besagt zum Beispiel, dass es keinen Zusammenhang zwischen der Intensität des Photostroms und der Anzahl der herausgelösten Elektronen des Metalls gibt.\\
Dies wurde durch die in Aufgabe 2 gemessenen Werte eindeutig widerlegt. Je höher die Intensität des Photostroms, desto höher ist auch das Maß der emittierten Elektronen.\\
Auch besagt die Wellentheorie, dass die kinetische Energie der Photonen direkt von der Intensität des einfallenden Lichts abhängig ist. Wenn ein Elektron von einer oszillierenden Schwingung angeregt werden würde, müsste es das Metall verlassen, wenn eine kritische Amplitude überschritten wäre. Das Elektron hätte also demnach eine gewisse kinetische Energie. Des weiteren hat diese Schwingung die Folge, dass es erst kurze Zeit nach dem Einschalten einen Messbaren Elektronenstrom gibt. Unseren Beobachtungen zufolge, war nach dem Einschalten instantan ein konstanter Elektronenstrom vorhanden.\\
Die Erkenntnisse der Photoemission belegen jedoch, dass in Wirklichkeit die Frequenz des Lichts einen direkten Einfluss auf die kinetische Energie der Elektronen hat.\\
Auch die in Aufgabe 3 erhaltenen Messwerte liefern einen eindeutigen Zusammenhang zwischen Frequenz und Bremsspannung, was der klassischen Ansicht nach unmöglich ist.


\section{Fazit}
Im Rahmen dieses Versuchs konnten wir unser Wissen über den Photoeffekt intensivieren und auch Experimentell die Vorhersagen der Wellentheorie mit denen der Quantenhypothese vergleichen. Wir können eindeutig die Inkorrektheit der Wellentheorie bestätigen. Ob die Quantenhypothese vollständig richtig ist können wir aufgrund der Ungenauigkeit der Messungen nicht sagen. Es waren sehr große systematische Fehler vorhanden und somit liefert die Auswertung der Daten nur ungenaue Werte. 
\section{Quellenangabe}
\begin{itemize}
\item GPII-Skript
\item Platzskript
\item Physik für Ingenieure von Ekbert Hering, Springer Verlag, S.477 ff.
\item Abbildung 1: \(http://www.gesundheit.de/sites/default/files/images/roche/pics/a22297.000-1_big.gif\)
\item Abbildung 2:  \(http://www.gesundheit.de/sites/default/files/images/roche/pics/a22297.000-1_big.gif\)
\end{itemize}


\end{document}