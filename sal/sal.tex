\documentclass{article}
\usepackage{indentfirst}
\usepackage{lmodern}
\usepackage[utf8]{inputenc}
\usepackage[T1]{fontenc}
\usepackage[ngerman]{babel}
\usepackage{amssymb,amstext,amsmath}
\usepackage{graphicx}
\usepackage{dsfont}
\usepackage{amsfonts}
\usepackage{graphics}
\usepackage{float}
\usepackage{cite}
\usepackage{url}
\usepackage{tabularx}
\usepackage{capt-of}
\usepackage{multirow}
\usepackage{hyperref}

\title{Schallwellen}
\author{Alexander Heinisch, Dominik Wille}
\begin{document}
\maketitle

{\begin{center}
\begin{minipage}{\linewidth}
\centering
\makebox[0cm]{\includegraphics[width=7cm]{Bilder/sal0}}
\label{wtd}
\end{minipage}
\end{center}

\vspace{7cm}
\noindent
\begin{center}
\begin{tabular}{r l}
Tutor & Florian Brünig\\
Durchführung & 12. Juni 2013 von 14-18 Uhr \\

E-Mail Dominik & dominik.wille@fu-berlin.de \\
E-Mail Alexander & Matthias.Heinisch@gmx.de \\
\end{tabular}
\end{center}

\newpage
\tableofcontents
\newpage

\section{Physikalische Grundlagen}
\subsection{Schallwellen und Ausbreitungsgeschwindigkeit}
Tritt eine zunächst lokale Erregung in einem ausgedehnten, elastischen Medium auf, entsteht eine Welle, welche sich räumlich ausbreitet. Die Geschwindigkeit, mit der sich diese Erregung ausdehnt, nennt man Phasen-, bzw. Schallgeschwindigkeit und hängt von der Rückstellkraft und der Trägheit des Mediums ab. Es gilt:

\begin{equation}
\label{c}
c=\sqrt{\frac{D}{\rho}}
\end{equation}

Mit der Rückstellkonstanten D und der Dichte \(\rho\).\\
Nimmt man nun einen Festkörper als Medium, besitzt in diesem jedes Volumenelement eine fest definierte Ruheposition. Nun unterscheidet man zwischen zwei Arten von Wellen in diesem Körper. Die Erste ist die longitudinale Dichtewelle, welche in Ausbreitungsrichtung schwingt. Des Weiteren gibt es die transversale Scherwelle. Sie schwingt im Gegensatz zur Longitudinalwelle senkrecht zur Ausbreitungsrichtung, ist nicht an ein Medium gebunden und kann polarisiert werden (z.B. Licht).In Gasen und Flüssigkeiten treten nur Dichtewellen auf, wodurch die Rückstellkonstante durch das Kompressionsmodul K ausgedrückt wird.\\
Da zum einen die Periodendauer von Schallwellen in Gasen relativ kurz ist, und zum anderen Gase eine geringe Wärmeleitfähigkeit besitzen, findet kein Energieaustausch zwischen den einzelnen Teilchen statt. Damit gilt die Poisson-Gleichung:
\begin{equation}
\label{rho}
\rho \cdot V^{\kappa} = const.
\end{equation}
welche durch Ableiten einen Ausdruck für die Kompressionsrate K liefert:
\begin{equation}
\label{K}
D = K = V \frac{d\rho}{dV}=-\kappa \rho
\end{equation}
mit dem Isentropenindex 
\begin{equation}
\label{kappa}
\kappa = \frac{c_{p}}{c_{v}}
\end{equation}
Daraus folgt für c:
\begin{equation}
c=\sqrt{\kappa\frac{p}{\rho}}
\end{equation}
Mit \(PV = NRT\) erhält man
\begin{equation}
c(T) = \sqrt{\kappa\frac{RT}{M}} \label{c(T)}
\end{equation}
Die Schallgeschwindigkeit ist also nur Temperaturabhängig.

\newpage
\subsection{Stehende Welle}
Durch Reflexion und Interferenz an den Grenzen eines beschränkten Volumens, entsteht eine Folge stationärer Schwingungszustände.Es bildet sich also eine stehende Welle, wenn die Wellenlänge \(\lambda\) in einem bestimmten Verhältnis zur Resonatorlänge \(l\) steht.\\
Ist der Resonator nun einseitig abgeschlossen, gilt die Relation:
\begin{equation}
l=\left(n-\frac{1}{2}\right)\frac{\lambda}{2} \label{geschlossen}
\end{equation}
Im beidseitig abgeschlossenen und beidseitig offenen Fall, gilt:
\begin{equation}
l=n\cdot\frac{\lambda}{2} \label{offen}
\end{equation}
In der folgenden Abbildung werden diese Formeln noch einmal graphisch dargestellt.
{\begin{center}
\begin{minipage}{\linewidth}
\centering
\makebox[0cm]{\includegraphics[width=\textwidth]{Bilder/sal1}}
\captionof{figure}{Darstellung von Wellen in a) unbegrenzten Medien, b) einseitig abgeschlossenem Resonator, c) und beidseitig abgeschlossenem Resonator}
\label{welle}
\end{minipage}
\end{center}

Kennen wir nun die Wellenlänge und die Frequenz , ergibt sich daraus die Fundamentalbeziehung der Schallgeschwindigkeit aus:

\begin{equation}
c=\lambda\cdot f
\end{equation}

Im Versuch werden wir bei der Untersuchung einer Gassäule in einem Rohr nur Longitudinalwellen beobachten. Befestigen wir den Stab in der Mitte, wird an einem Ende ein Phasensprung von \(\pi\) auftreten und man kann die Grundschwingung von n=1 beobachten.\\
Befindet sich der Stab hingegen zu einem viertel seiner Länge vom Rand entfernt, ist die ''erste Harmonische'' mit n=2 zu beobachten.\\


\newpage
\section{Aufgabenstellung}
\subsection*{Aufgabe 1}
Messung der Schallgeschwindigkeit in Luft durch Laufzeitmessung.
\subsection*{Aufgabe 2}
Beobachtung der Resonanzen einer Luftsäule mit abgeschlossenem bzw. offenem Ende durch Variation der Anregungsfrequenz. Berechnung der Schallgeschwindigkeit und des Verhältnisses der spezifischen Wärme \(\frac{c_{p}}{c_{v}}=\kappa\) von Luft (Isentropenindex, Adiabatenkoeffizient).
\subsection*{Aufgabe 3}
Bestimmung der Schallgeschwindigkeit in Metallen aus der Laufzeit bzw. der Grundschwingungsfrequenz für zwei verschiedene Einspannungen des Stabes. Berechnung des Elastizitätsmoduls des Metalls.
\subsection*{Ergänzende Frage}
Warum werden höhere Frequenzen stärker gedämpft als niedrige?

\newpage

%\section{Auswertung}
\section{Aufgabe 1}
Im Rahmen dieser Aufgabe soll die Schallgeschwindigkeit in Luft direkt gemessen werden. Indem die Zeit gemessen wird, die ein Signal braucht um eine gewisse Strecke zurückzulegen.
\subsection{Aufbau}
Der Versuchsaufbau besteht aus einer Klatsche, die bei Betätigung einen akustischen Impuls abgibt. Um die Zeit zu messen, die das Signal bis zu einem bestimmten Punkt benötigt wird bei der Betätigung der Klatsche zusätzlich ein elektronisches Signal abgeben, dieses wird vom Oszilloskop als \textit{Trigger} benutzt. An dem Ende der Messstrecke wird ein Mikrophon aufgebaut, dass an das Oszilloskop angeschlossen wird. Auf dem Oszilloskop kann dann die Zeit zwischen dem Triggersignal und einem Ausschlag durch den vom Mikrophon aufgenommen Schall bestimmt werden. Verwendet wird ein Digitales Oszilloskop, dass die Daten direkt an den Computer überträgt, damit sie gut gespeichert werden können.
\subsection{Fehlerabschätzung}
Der Fehler der Messung setzt sich aus einer Ungenauigkeit der Streckenmessung und einem Fehler der Zeitmessung zusammen. 

Bei der Streckenmessung kommt zu den Messungenauigkeiten noch ein systematischer Fehler, da nicht absehbar ist, wo sich das Mikrophon genau in der Apparatur befindet, außerdem ist auch nicht klar, was als Ausgangspunkt des akustischen Signals gelten soll. Um das in der Rechnung zu berücksichtigen wird \(s\) mit einem systematischen Fehler \(s_{sys}\) addiert.

Auch bei der Messung der Zeit ist ein systematische Fehler vorhanden dessen Ursache nicht genau klar ist, da kaum etwas über die Geräte bekannt ist. Eine Mögliche Ursache ist, dass eventuell das Trigger- oder das akustische Signal nicht Verzögerungsfrei weitergeleitet wird. Daher wird auch hier ein \(t_{sys}\) addiert.
\begin{align}
s_{mess} &= s + s_{sys}\\
t_{mess} &= t + t_{sys}
\end{align}

Die statistischen sind schwer abzuschätzen. Es ist unter Anderem nicht immer klar ist, wann genau das akustische Signal detektiert wurde. Ein Fehlerwert dennoch nur schätzbar, daher werden im Anschluss die Fehler aus der Streuung herangezogen. 

\subsection{Messwerte}
Eine Messung lieferte zu jeder gewählten Distanz eine v-t-Kurve, in der die Zeit zwischen Triggersignal und akustischem Signal abgelesen werden kann. Die Werte sind dabei nur mit dem Auge ermittelt worden. Um sich das besser vorstellen hier eine Beispielkurve:

\begin{center}
\begin{minipage}{\linewidth}
\centering
\makebox[0cm]{\includegraphics[width=\textwidth]{graphen/a1/100_1.png}}
\captionof{figure}{Beispielhafte Ausgabe des Digitaloszilloskops bei s = 1m}
\end{minipage}
\label{bsp}
\end{center}
Die Zeit wurde bei diesem Beispiel geschätzt auf \(t = 3,9\, ms\). Bei den weiteren Messungen wurde identisch verfahren.
%Aufgenommen wurden zwei Messreihen, deswegen sind zu jeder Strecke zwei Zeitmessungen vorhanden. 
Zusammengefasst wurden folgende Werte ermittelt:
\begin{center}
\begin{tabular}{c|c}
Strecke \(s\) in \(cm\) & Zeit \(t\) in \(ms\) \\\hline
\(100\) & \( 3.9\) \\ 
\(100\) & \( 4\) \\ 
\(110\) & \( 4.1\) \\ 
\(110\) & \( 4.1\) \\ 
\(120\) & \( 4.7\) \\ 
\(120\) & \( 4.5\) \\ 
\(130\) & \( 4.8\) \\ 
\(130\) & \( 5.1\) \\ 
\(140\) & \( 4.9\) \\ 
\(140\) & \( 5.1\) \\ 
\(150\) & \( 5.2\) \\ 
\(150\) & \( 5.4\) \\ 
\(160\) & \( 5.5\) \\ 
\(160\) & \( 5.7\) \\ 
\(170\) & \( 5.8\) \\ 
\(170\) & \( 5.9\) \\ 
\(180\) & \( 5.9\) \\ 
\(180\) & \( 6\) \\ 
\(190\) & \( 6.7\) \\ 
\(190\) & \( 6.1\) \\
\end{tabular}
\captionof{table}{Abgelesene Zeiten für verschiedene Strecken}
\end{center}
\subsection{Graphen}
Um diese Messwerte auszuwerten werden sie zunächst in ein s-t-Diagramm eingetragen und anschließend eine Ausgleichsgerade mit einem Fehler aus der Streuung eingefügt:
\begin{center}
\begin{minipage}{\linewidth}
\centering
\makebox[0cm]{\includegraphics[width=\textwidth]{graphen/gnuplot/a1}}
\captionof{figure}{Beispielhafte Ausgabe des Digitaloszilloskops bei s = 1m}
\end{minipage}
\label{bsp}
\end{center}
Um die Plausibilität der Ausgleichsgeraden zu überprüfen wurden die Fehler geschätzt:
\begin{align}
\Delta t & = 30\, ms \notag \\ 
\Delta s & = 2\, cm \notag
\end{align}
Die lineare Regression wurde mit einer Funktion \(s(t) = a + b \cdot t\) durchgeführt und ergab folgende Parameter:
\begin{align}
a &= \left(-430 \pm 93 \right)\, mm \notag \\
b &= \left( 363 \pm 18 \right)\, \frac{m}{s} \notag
\end{align}
\subsection{Auswertung}
Die Geschwindigkeit ist allgemein definiert als:
\begin{align}
v = \frac{ds}{dt}
\end{align}
angewendet auf die Funktion der linearen Regression ergibt für die Schallgeschwindigkeit:
\begin{align}
c &= \frac{d}{dt} \left(a + b \cdot \left(t - t_{sys} \right) - s_{sys} \right)  \label{drv}\\
\Rightarrow c &= b \\
\Rightarrow c &=  \left( 363 \pm 18 \right)\, \frac{m}{s} \notag
\end{align}
\subsection{Fazit und Vergleich}
Der Versuch war sehr gut durchführbar, die Streuung der Messwerte ist zwar recht groß, jedoch ist zu sehen, dass alle Messwerte mit den geschätzten Fehlern die Ausgleichsgerade schneiden. Die ermittelte Schallgeschwindigkeit ist:
\begin{align}
c = \left( 360 \pm 20 \right)\, \frac{m}{s} \notag
\end{align}
Dieses Ergebnis ist identisch zum Literaturwert, dieser liegt bei Zimmertemperatur bei ca. \(343\, \frac{m}{s}\).

Interessant ist, dass der systematische Fehler \(a\) mit ca. \(-43\, cm\) extrem groß ist. Der Grund dafür, kann nur sein, dass auch \(t\) einen signifikanten systematischen Fehler enthält da die Strecke nicht so ungenau gemessen wurde.  Da \(a\) negativ ist muss also auch \(t_{sys}\) negativ sein. Das kann nur heißen, dass das Triggersignal zu spät übermittelt wird.
\newpage
\section{Aufgabe 2}
In dieser Aufgabe soll die Schallgeschwindigkeit und der Isentropenindex bzw. Adiabatenkoeffizient über die Resonanzfrequnzen in einem Rohr bestimmt werden.
\subsection{Aufbau}
Vorhanden ist ein Rohr, dass auf der einen Seite einen Lautsprecher und auf der anderen Seite eine Öffnung hat, die man jedoch mit einem Brettchen verschließen kann. Der Lautsprecher wird mit Bananensteckerkabeln mit dem Funktionsgenerator verbunden um bestimmte Schallwellen im Rohr zu erzeugen. Gemessen wird qualitativ die Lautstärke an einer kleinen Öffnung am Rohr indem ein Mikrophon an ein Wechselspannungsmessgerät angeschlossen wird.
\subsection{Durchführung}

\subsection{Fehlerabschätzung}
\subsection{Gegebenes}
\subsection{Messwerte und Graphen}
\subsection{Auswertung}
\subsection{Fazit und Vergleich}
\newpage
\section{Aufgabe 3}
Im Rahmen dieser Aufgabe wird die Schallgeschwindigkeit in einem Metall bestimmt, indem ein Metallstab so eingespannt wird, das nur eine Schwingmode auftritt.
\subsection{Aufbau}
Der Aufbau gestaltet sich in diesem Fall relativ einfach. Die Metallstäbe werden so in Stative eingespannt, dass entweder nur die \(n=1\)- oder die \(n=2\)-Mode angeregt werden kann. Für die \(n=1\)-Mode ist das die Mitte des Stabs und für die \(n=2\)-Mode die Punkte bei \(\frac{1}{4}\) und \(\frac{3}{4}\). Um die Schwingfrequenz zu messen wird ein Mikrophon an den Stab postiert und an das Computer-Oszilloskop angeschlossen.
\subsection{Fehlerabschätzung}
\subsection{Gegebenes}
Gegeben sind die Längen und dichten der Stäbe mit:
\begin{center}
\begin{tabular}{c|c}
Größe & Wert\\\hline
\(l_{Stahl}\) & \((149,95 \pm 0,1)\,cm\) \\
\(l_{Messing}\) & \((146,35 \pm 0,1)\,cm\) \\
\(\rho_{Stahl}\) & \((7,5 \pm 0,3)\,\cdot 1000\frac{kg}{m^3}\) \\
\(\rho_{Messing}\) & \((8,4 \pm 0,1)\,\cdot 1000\frac{kg}{m^3}\)
\end{tabular}
\captionof{table}{Gegebene Daten der Metallstäbe}
\end{center}
\subsection{Kurven und Messwerte} Gemessen wurde die Frequenz indem die Schwingungen pro Zeitintervall ermittelt wurden. Beispielhaft sieht das so aus:
\begin{center}
\noindent
\begin{minipage}{\linewidth}
\centering
\makebox[0cm]{\includegraphics[width=\textwidth]{graphen/a3/Messing_1_1}}
\captionof{figure}{Frequnz Messingstab in der \(n=1\)-Mode}
\end{minipage}
\end{center}
\begin{equation}
\Rightarrow f = \frac{n}{\Delta T} \approx \frac{7}{6ms} = 1166\, Hz \notag
\end{equation}
So wurden folgende Frequenzen ermittelt:
\begin{center}
\begin{tabular}{c|c|c}
Mode \(n\) & \(f\, [Hz]\) Messing & \(f\, [Hz]\) Stahl \\\hline
1 & 1200 & 1633 \\
1 & 1166 & 1600 \\
2 & 2308 & 3269 \\
2 & 2353 & 3269
\end{tabular}
\end{center}
\subsection{Graphen}
Um diese Auswerten zu können werden diese geplottet und mit einer Ursprungsgeraden gefittet.
\begin{center}
\noindent
\begin{minipage}{\linewidth}
\centering
\makebox[0cm]{\includegraphics[width=\textwidth]{graphen/gnuplot/a3m}}
\captionof{figure}{Frequnzen beim Messingstab}
\end{minipage}
\end{center}
\begin{center}
\noindent
\begin{minipage}{\linewidth}
\centering
\makebox[0cm]{\includegraphics[width=\textwidth]{graphen/gnuplot/a3s}}
\captionof{figure}{Frequnzen beim Stahlstab}
\end{minipage}
\end{center}
Die lineare Regression ergab folgende Steigungen:
\begin{align}
b_{Messing} = (1168,8 \pm 8,4)\, Hz \notag \\
b_{Stahl} = (1630,9 \pm 5,9)\, Hz \notag
\end{align}
\subsection{Auswertung}
Da die Moden der Stäbe offenen Enden entsprechen gilt wie in Aufgabe 2 auch hier:
\begin{equation}
f(n) = n \cdot \frac{c}{2l}
\end{equation}
Daher kann die Steigung identifiziert werden um \(c\) zu bestimmen:
\begin{align}
b &= \frac{c}{2l}\\
\Rightarrow c &= b \cdot 2l\\
\Delta c &= c \cdot \sqrt{\left( \frac{\Delta b}{b} \right)^2+ \left( \frac{\Delta l}{l} \right)^2}\\
c_{Messing} &= (3421 \pm 25)\, \frac{m}{s} \notag \\
c_{Stahl} &=  (4891 \pm 18)\, \frac{m}{s} \notag
\end{align}
Um daraus das Elastizitätsmodul \(E\) zu errechnen \eqref{c} verwendet, für einen Festkörper gilt \(D = E\):
\begin{align}
c^2 &= \frac{E}{\rho} \\
\Rightarrow E &= c^2 \cdot \rho \\
\Delta E &= E \cdot \sqrt{\left(\frac{2\Delta c}{c}\right)^2+\left(\frac{\Delta \rho}{\rho}\right)^2 }\\
E_{Messing} &= (98,3 \pm 1,6) \, \frac{kN}{mm^2} \notag \\
E_{Stahl} &= (179,4 \pm 7,3) \, \frac{kN}{mm^2} \notag
\end{align}
\subsection{Fazit und Vergleich}
Auch dieser Versuch war sehr einfach durchführbar. Die Messung war sogar relativ genau und qualitativ stimmt die Messung mit den Erwartungen überein, da Die Werte zu einer Ursprungsgeraden passen. Ein Vergleich mit Literaturwerten ist schwierig, da sowohl die Schallgeschwindigkeiten, als auch die Elastizitätsmodule stark von der genauen Legierung abhängen. Eine Gegenüberstellung soll dennoch nicht ausbleiben:
\begin{center}
\begin{tabular}{c|c|c|c}
Größe & Messung & Literatur\\\hline
\(c_{Messing}\) & \((3420 \pm 30)\, \frac{m}{s}\) & \(3500\, \frac{m}{s}\)\\
\(c_{Stahl}\) & \((4890 \pm 20)\, \frac{m}{s}\) & \(4900\, \frac{m}{s}\)\\
\(E_{Messing}\) & \((98 \pm 2) \, \frac{kN}{mm^2}\) & \(78-123 \, \frac{kN}{mm^2}\) \\
\(E_{Stahl}\) & \((179 \pm 8) \, \frac{kN}{mm^2}\) & \(195-210 \, \frac{kN}{mm^2}\)
\end{tabular}
\end{center}
Die Ermittelten Werte sind also sehr nahe an Literaturwerten ähnlicher Metalle. Es ist somit nicht davon auszugehen, dass die Messung nicht erfolgreich verlaufen ist.

\section{zu der ergänzenden Frage}
Eine mögliche Ursache für die größere Dämpfung bei großen Frequenzen ist, dass sich durch durch die schnellere Oszillation der Teilchen auch mehr Reibung zwischen den Teilchen auftritt.

\section{Gesamtfazit}
Alles in allem waren alle drei Aufgaben sehr gut durchführbar. Die Schallgeschwindigkeit konnte sowohl in Luft als auch in Metallen erfolgreich bestimmt werden. Dabei lieferten auch verschiedene Messmethoden für die Schallgeschwindigkeit in Luft gleiche Ergebnisse.

\section{Quellenangabe}
\begin{itemize}
\item Alle Literaturwerte wurden aus der Formelsammlung des Schulbuchverlags entnommen.
\item Als Orientierung für die Theoretischen Grundlagen wurde das GP2-Skript herangezogen.
\end{itemize}
\vspace{7.0cm}

\begin{tabularx}{\textwidth}[b]{p{5cm} X p{5cm}} \cline{1-1} \cline{3-3}
Datum, Dominik Wille & & Datum, Alexander Heinisch
\end{tabularx}
\end{document}