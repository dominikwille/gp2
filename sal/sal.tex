\documentclass{article}
\usepackage{indentfirst}
\usepackage{lmodern}
\usepackage[utf8]{inputenc}
\usepackage[T1]{fontenc}
\usepackage[ngerman]{babel}
\usepackage{amssymb,amstext,amsmath}
\usepackage{graphicx}
\usepackage{dsfont}
\usepackage{amsfonts}
\usepackage{graphics}
\usepackage{float}
\usepackage{cite}
\usepackage{url}
\usepackage{tabularx}
\usepackage{capt-of}

\title{Schallwellen}
\author{Alexander Heinisch, Dominik Wille}
\begin{document}
\maketitle

{\begin{center}
\begin{minipage}{\linewidth}
\centering
\makebox[0cm]{\includegraphics[width=7cm]{bilder/sal0}}
\label{wtd}
\end{minipage}
\end{center}

\vspace{7cm}
\noindent
\begin{center}
\begin{tabular}{r l}
Tutor & Brünig\\
Durchführung & 12. Juni 2013 von 14-18 Uhr \\

E-Mail Dominik & dominik.wille@fu-berlin.de \\
E-Mail Alexander & Matthias.Heinisch@gmx.de \\
\end{tabular}
\end{center}

\newpage
\tableofcontents
\newpage

\section{Physikalische Grundlagen}

\end{document}