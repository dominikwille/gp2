\newpage
\section{Aufgabe 2}
In dieser Aufgabe soll die Schallgeschwindigkeit und der Isentropenindex bzw. Adiabatenkoeffizient über die Resonanzfrequenzen in einem Rohr bestimmt werden. Dabei ist das Rohr in einem Fall offen und im anderen geschlossen.
\subsection{Aufbau}
Vorhanden ist ein Rohr, dass auf der einen Seite einen Lautsprecher und auf der anderen Seite eine Öffnung hat, die man jedoch mit einem Brettchen verschließen kann. Der Lautsprecher wird mit Bananensteckerkabeln mit dem Funktionsgenerator verbunden um bestimmte Schallwellen im Rohr zu erzeugen. Gemessen wird qualitativ die Lautstärke an einer kleinen Öffnung am Rohr indem ein Mikrophon an ein Wechselspannungsmessgerät angeschlossen wird.
\subsection{Durchführung}
Es werden sinusförmige Wechselströme auf den Lautsprecher gegeben und die Frequenz am Funktionsgenerator so Eingestellt, dass ein Maximum der Lautstärke gemessen wird.
\subsection{Fehlerabschätzung}
Leider ist die Messung der Frequenz sehr ungenau, da weder der Funktionsgenerator eine genaue Frequenz anzeigt, noch die Frequenz sehr exakt eingestellt werden konnte, da die Lautstärke in einem gewissen Bereich eher Konstant war. Der Fehler der Frequenz wird geschätzt auf:
\begin{equation}
\Delta f = 7\% \notag
\end{equation}
\newpage
\subsection{Messwerte und Graphen}
Ermittelt wurden folgende Werte für Resonanzfrequenzen:

\begin{center}
\begin{tabular}{c|cc}
\multirow{2}{*}{Ordung \(n\)} & \multicolumn{2}{c}{Resonanzfrequnzen \(f\)}\\
 & Messreihe 1 & Messreihe 2 \\\hline
\(1\) & \(73\) & \(75\) \\ 
\(2\) & \(227\) & \(217\) \\ 
\(3\) & \(416\) & \(457\) \\ 
\(4\) & \(620\) & \(616\) \\ 
\(5\) & \(782\) & \(786\) \\ 
\(6\) & \(958\) & \(955\) \\ 
\(7\) & \(1114\) & \(1123\) \\ 
\(8\) & \(1286\) & \(1292\) \\ 
\(9\) & \(1415\) & \(1420\) \\ 
\(10\) & \(1573\) & \(1580\) \\ 
\(11\) & \(1730\) & \(1730\) \\ 
\(12\) & \(1887\) & \(1883\) \\ 
\(13\) & \(2068\) & \(2064\) \\
\end{tabular}
\captionof{table}{Messwerte der geschlossenen Röhre}
\vspace{1cm}
\begin{tabular}{c|c}
Ordnung \(n\) & Resonanzfrequnzen \(f\) \\\hline
\(1\) & \(60\) \\ 
\(2\) & \(153\) \\ 
\(3\) & \(278\) \\ 
\(4\) & \(350\) \\ 
\(5\) & \(395\) \\ 
\(6\) & \(528\) \\ 
\(7\) & \(683\) \\ 
\(8\) & \(866\) \\ 
\(9\) & \(1030\) \\ 
\(10\) & \(1205\) \\ 
\(11\) & \(1471\) \\ 
\(12\) & \(1636\) \\ 
\(13\) & \(1784\) \\ 
\(14\) & \(1955\) \\ 
\(15\) & \(2243\) \\
\end{tabular}
\captionof{table}{Messwerte der offenen Röhre}
\end{center}
\subsection{Auswertung}
Um die Daten auswerten zu können werden sie nun Graphisch dargestellt
\subsection{Fazit und Vergleich}