\newpage
\section{Aufgabe 2}
In dieser Aufgabe soll die Schallgeschwindigkeit und der Isentropenindex bzw. Adiabatenkoeffizient über die Resonanzfrequnzen in einem Rohr bestimmt werden.
\subsection{Aufbau}
Vorhanden ist ein Rohr, dass auf der einen Seite einen Lautsprecher und auf der anderen Seite eine Öffnung hat, die man jedoch mit einem Brettchen verschließen kann. Der Lautsprecher wird mit Bananensteckerkabeln mit dem Funktionsgenerator verbunden um bestimmte Schallwellen im Rohr zu erzeugen. Gemessen wird qualitativ die Lautstärke an einer kleinen Öffnung am Rohr indem ein Mikrophon an ein Wechselspannungsmessgerät angeschlossen wird.
\subsection{Durchführung}

\subsection{Fehlerabschätzung}
\subsection{Gegebenes}
\subsection{Messwerte und Graphen}
\subsection{Auswertung}
\subsection{Fazit und Vergleich}