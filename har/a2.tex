\section{Exercise 2-4}
To investigate driven oscillations a electric motor is used to drive the oscillation.
The position of the flywheel is again determined by the CASSY system and recorded
on the PC. To get the data for exercise 3 and 4 also the phase shift and the 
transient responses are observed qualitively.

\subsection{Observation}
\subsubsection{Exercise 2}
To determine the amplitude \(A\) as a function of the drive frequency \(\Omega\) 
the interval of drive frequencies is chosen so that it contains the resonance frequency 
\(\omega_1\) and the values near the resonance frequency.
  
\subsubsection{Exercise 3}
It is very clear that in this experiment the observes phase shift qualitively absolutely fits
with the theoretical function. For very small \(\Omega\) the phase shift is \(\phi
= 0\) for high frequencies the phase shift is \(\phi = - \pi\) and near the resonance
frequency it is \(\phi = -\pi/2 \).
\subsubsection{Exercise 4}
