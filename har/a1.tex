\newpage
\section{Exercise 1}
In this task the oscillator will be deflected by hand to an amplitude near the maximum. The position resp. Angular of the flywheel Is measured and plotted with CASSY-Lab. 
\subsection{Observation}
The measured curves fit nearly perfectly the expectations of a exponential decreasing sin function. To process the data The envelope function will is fitted with a \(f(x)=A*e^{-x/B}+C\) function. The frequency  is measured just by counting the periods per time interval.
\subsection{Errors}
The equipment does not provide data which is totally free of systematic errors. There is a lot of friction which is not only dependent on the wheels velocity. But the systematic errors are not so significant that the data can not be evaluated.

It is assumed that the time is measured relatively correct \(\Delta t = 0.01\, s\). 
\subsection{Measured Data}
\begin{center}
\begin{tabular}{c|cccc}
Position & \(B [s]\) & \(N [1]\) & \(t_1 [s]\) & \(t_2 [s]\) \\ \hline
3 & \(5.12\) & \(6\) & \(1.72 \pm 0.001\) & \(13.6 \pm 0.001\) \\ 
3 & \(5.1\) & \(6\) & \(1.69 \pm 0.001\) & \(13.68 \pm 0.001\) \\ 
3 & \(5.13\) & \(6\) & \(1.65 \pm 0.001\) & \(13.62 \pm 0.001\) \\ 
2 & \(5.41\) & \(6\) & \(13.90 \pm 0.001\) & \(1.68 \pm 0.001\) \\ 
2 & \(5.34\) & \(6\) & \(1.70 \pm 0.001\) & \(13.92 \pm 0.001\) \\ 
2 & \(5.4\) & \(6\) & \(1.73 \pm 0.001\) & \(13.94 \pm 0.001\) \\ 
1 & \(27.13\) & \(20\) & \(42.01 \pm 0.001\) & \(4.22 \pm 0.001\) \\ 
1 & \(26.95\) & \(20\) & \(4.26 \pm 0.001\) & \(42.13 \pm 0.001\) \\ 
1 & \(27.23\) & \(20\) & \(14.83 \pm 0.001\) & \(52.64 \pm 0.001\) \\ 
0 & \(69.13\) & \(20\) & \(50.82 \pm 0.001\) & \(13.16 \pm 0.001\) \\ 
0 & \(68.89\) & \(20\) & \(13.08 \pm 0.001\) & \(50.73 \pm 0.001\) \\ 
0 & \(69.23\) & \(20\) & \(13,00 \pm 0.001\) & \(50.65 \pm 0.001\) \\ 
\end{tabular}
\captionof{table}{measured values for free damped oscillations}
\end{center}
\subsection{Evaluation}
To determine the damping constant \(\delta\) the formula \eqref{x(t)_1} can be compared to the fitting function \(f(x)=A*e^{-x/B}+C\) and gives:
\begin{align}
\delta = \frac{1}{B}
\end{align}
Moreover the eigenfrequency \(\omega_0\) can be determined with \eqref{w1} as:
\begin{align}
\omega_0 = \sqrt{\omega_1^2 + \delta^2}
\end{align}