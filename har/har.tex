\documentclass{article}
%\usepackage{indentfirst}
\usepackage{lmodern}
\usepackage[utf8]{inputenc}
\usepackage[T1]{fontenc}
%\usepackage[ngerman]{babel}
\usepackage[english]{babel}
\usepackage{amssymb,amstext,amsmath}
\usepackage{graphicx}
\usepackage{sectsty}
\usepackage{multirow}
\usepackage{dsfont}
\usepackage{amsfonts}
\usepackage{graphics}
\usepackage{float}
\usepackage{dsfont}
\usepackage{hyperref}
%\usepackage{titlesec}
%\titleformat{\section}[hang]{\bfseries\sffamily \LARGE}{\thesection.}{.5em}{}[]
%\titleformat{\subsection}[hang]{\bfseries\sffamily \large}{\thesubsection.}{.5em}{}[]
%\titlespacing{\subsection}{\parindent}{*2}{\wordsep}
%\allsectionsfont{\bfseries\sffamily}

%the needed packages
\usepackage[absolute]{textpos}
\usepackage[texcoord]{eso-pic}

%adjust the TPHorizModule and TPHorizModule units to the displayed mm %grid
\TPGrid{210}{297}

%puts a graphic at the absolute position described by the grid
%#1 x, #2 y, #3 width, #4 height, #5 graphic
\newcommand\putpic[5]{%
        \begin{textblock}{#3}(#1,#2)
  \includegraphics[width=#3\TPHorizModule]{#4}
     \end{textblock}
}

%\renewcommand*{\size@chapter}{\Huge}

\usepackage{cite}
\usepackage{url}
 
\title{Harmonic Oscillations}
\author{Alexander Heinisch, Dominik Wille}
\begin{document}
\maketitle
\vspace{12cm}
\noindent
\begin{tabular}{l l} 
Execution & June 26, 2013 \\
Tutor & Ivelina Zaharieva  \\
E-Mail Dominik & dominik.wille@fu-berlin.de \\
E-Mail Alexander & matthias.heinisch@gmx.de \\
 \end{tabular}

\newpage
\tableofcontents
\newpage
\section{Introduction}
In this document will be discussed how a real harmonic oscillating system moves. Therefore the classic mechanic will be used to determine the equation of motion. Later on the theoretical equation will be compared to the measured one.
\section{Physical Principles}
To develop a equation of motion the Newtonian mechanics will be used. In principle there is a second order differential equation given by Newtons 2nd Law where \(a\) is the acceleration and \(m\) the mass of the moving part. In mathematical terms acceleration just means the second derivative after the time of the position \(x\). The discussed system has only one degree of freedom so all forces \(F\) and the position \(x\) are just scalars.
\begin{align}
F &= m \cdot a\notag\\
 &= m \cdot \frac{d^2}{dt^2}x\notag\\
 &= m \cdot \ddot x
\label{2nd_law}
\end{align}
\subsection{Determining the Forces}
\subsubsection{The Spring Force}
Robert Hook first described the Force of a spring by its extension. He assumed that for small amplitudes the force will just be linear in the displacement. The spring force therefore is
\begin{equation}
F_s = -D \cdot \left( x - x_0 \right) \label{gen_spring}
\end{equation}
To make it more simple the coordinate system will be chosen so that \(x_o = 0 \). 
\begin{equation}
F_s = -D \cdot  x
\label{spring}
\end{equation}
\subsubsection{Friction}
To get a good approximation for the real motion of the system it is absolutely necessary to enter a friction force. A simple way to describe a friction force is to assume that it is linear in and only dependent on the velocity. The friction force will be
\begin{equation}
F_f = - k \cdot v
\notag
\end{equation}
A velocity is just the first derivative after the time of the position so
\begin{align}
F_f &= - k \frac{d}{dt} x \notag \\
 &= - k \dot x
\label{friction}
\end{align}
\subsection{Equation of Motion for free damped oscillations}
The force \(F\) that acts on the mass \(m\) is the sum of all forces.
\begin{align}
F &= F_s + F_f \notag 
\end{align}
With \eqref{spring} and \eqref{friction}
\begin{align}
F &= -D x - k \dot x \notag
\end{align}
So equation of motion \eqref{2nd_law} becomes
\begin{align}
m\ddot{x} &= -D x - k \dot{x} \notag\\
\Rightarrow 0 &= D x + k \dot{x} + m\ddot{x}
\label{eom}
\end{align}
The equation is a ordinary second order linear differential equation. In general all ordinary linear differential equations can be solved with the ansatz:
\begin{align}
x(t) &= ae^{\lambda t} \notag\\
\Rightarrow \dot{x}(t) &= a \lambda e^{\lambda t} \notag\\
\Rightarrow \ddot{x}(t) &= a \lambda^2 e^{\lambda t} \notag
\end{align}
insertion into \eqref{eom} gives:
\begin{align}
0 &= m a \lambda^2 e^{\lambda t} + k a \lambda e^{\lambda t} + D a e^{\lambda t} &\bigg\vert& \cdot \frac{1}{a} e^{- \lambda t} \notag\\
0 &= m \lambda^2 + k \lambda + D &\bigg\vert& \cdot \frac{1}{m} \notag\\
0 &= \lambda^2 + \frac{k}{m} \lambda + \frac{D}{m} && \notag
\end{align}
\begin{align}
\omega_0 := \sqrt{\frac{D}{m}} & & \delta := \frac{k}{2m} \notag
\end{align}
\begin{align}
0 &= \lambda^2 + 2\delta \lambda + \omega_0^2 \notag \\
\lambda_{1/2} &= - \delta\, \pm\, \sqrt{\delta^2-\omega_0^2} \notag \\
 &= - \delta\, \pm\, i\sqrt{\omega_0^2- \delta^2} \notag
\end{align}
The general solution is the linear combination of the two special solutions
\begin{align}
x(t) = a_1e^{\left(-\delta+i\sqrt{\omega_o^2-\delta^2} \right)t} + a_2e^{\left(-\delta-i\sqrt{\omega_o^2-\delta^2} \right)t}
\label{x_norm}
\end{align}
\(x\) is the position of the mass, and therefore a real number so \(x(t)=x^*(t)\). For this Experiment the oscillating case \(\delta < \omega_0\) is realized, so \(\sqrt{\omega_o^2-\delta^2}\) is real.
\begin{align}
x^*(t) = a_1^*e^{\left(-\delta-i\sqrt{\omega_o^2-\delta^2} \right)t} + a_2^*e^{\left(-\delta+i\sqrt{\omega_o^2-\delta^2} \right)t}
\label{x_konj}
\end{align}
Comparing \eqref{x_norm} and \eqref{x_konj} gives \(a_1^* = a_2\). In general every complex number can be written in the form \(Re^{i\varphi}\) so \(a_{1/2}\) can be too.
\begin{align}
a_1 &= \frac{A}{2} e^{i\varphi_0} && a_2 = \frac{A}{2} e^{-i\varphi_0} \notag
\end{align}
Inserting into \eqref{x_norm}
\begin{align}
x(t) &= \frac{A}{2} e^{i\varphi_0}e^{\left(-\delta+i\sqrt{\omega_o^2-\delta^2} \right)t} + \frac{A}{2} e^{-i\varphi_0}e^{\left(-\delta-i\sqrt{\omega_o^2-\delta^2} \right)t}\notag\\
 &= Ae^{-\delta t} \left(\frac{1}{2} e^{i\left(\varphi_0+\sqrt{\omega_o^2-\delta^2}\right)t} + \frac{1}{2} e^{-i\left(\varphi_0+\sqrt{\omega_o^2-\delta^2}\right)t} \right)\notag\\
  &= Ae^{-\delta t}\cos(\sqrt{\omega_o^2-\delta^2}t + \varphi_0)\notag
\end{align}
with
\begin{equation}
\omega_1 := \sqrt{\omega_o^2-\delta^2}
\label{w1}
\end{equation}
it gives the first form of the solution
\begin{align}
x(t) &= Ae^{-\delta t}\cos(\omega_1 t + \varphi_0)\label{x(t)_1}
\end{align}
There is an alternate form which can be derived with angle identity \(\cos(\alpha + \beta) = \cos \alpha \cos \beta - \sin \alpha \sin \beta \).
\begin{align}
x(t) &= Ae^{-\delta t}\left( \cos(\varphi_0)\cos(\omega_1 t) - \sin(\varphi_0)\sin(\omega_1 t) \right) \notag
\end{align}
With
\begin{align}
A_1 &= A\cos(\varphi_0) && A_2 = -A\sin(\varphi_0) \notag
\end{align}
it becomes the alternate form
\begin{align}
x(t) &= e^{-\delta t}\left(A_1 \cos(\omega_1 t) + A_2 \sin(\omega_1 t) \right) \label{x(t)_2}
\end{align}
\subsection{Equation of Motion for driven harmonic oscillations}
To investigate the resonance phenomena it is very useful to have a simple model in which the position of the equilibrium \(x_o\) moves harmonically.
\begin{align}
x_0(t) = A_0\cos \left( \Omega t\right)
\end{align}
So the spring force \eqref{gen_spring} becomes:
\begin{align}
F_s = -D\left(x -x_0(t) \right)
\end{align}
The equation of motion can be derived buy just adding all forces together:
\begin{align}
F &= F_s + F_f \notag\\
m \ddot{x} = F &= -D\left(x -x_0(t) \right) -k \dot{x} \notag \\
0 &= m \ddot{x}  + k \dot{x} + D\left(x -x_0(t) \right) \notag \\
0 &= m \ddot{x}  + k \dot{x} + Dx -DA_0\cos \left( \Omega t\right) \notag
\end{align}
With the definition \(DA_0 =: F_0\) the equation can easily be written as:
\begin{align}
\Rightarrow F_0 \cos \left( \Omega t \right) &= m \ddot{x} + k \dot{x} + Dx
\end{align}
It is absolutely clear that this is just a inhomogeneous version of the equation in the free damped oscillating case \eqref{eom}. So it is only necessary to derive a special solution for this equation. The general solution will be the sum of the general homogeneous solution \eqref{x(t)_1} and the special solution of the inhomogeneous equation.

The ansatz for this situation is chosen as:
\begin{align}
x(t) &= F \cdot e^{i(\Omega t + \phi)} \\
\dot{x}(t) &= i \Omega \cdot A_s \cdot e^{i(\Omega t + \phi)} \\
\ddot{x}(t) &= - \Omega^2 \cdot A_s \cdot e^{i(\Omega t + \phi)}
\end{align}
\section{Exercises}
\subsection*{Exercise 1}
Investigation of free undamped oscillations. Measuring the amplitude as a function of time. Determining the system's damping constant \(\delta\) and eigenfrequency \(\omega_0\).
\subsection*{Exercise 2}
Investigation of forced oscillations. Recording the amplitude as a function of time. Determining the damping constant and the eigenfrequency.
\subsection*{Exercise 3}
Qualitative observation of the phase shift \(\phi\) between exciter an oscillator as a function of the exciter's frequency \(\Omega\).
\subsection*{Exercise 4}
Observation of the swing behaviour for the resonance case and for an excitation frequency \(\Omega\) close to the resonance frequency \(\omega_0\).


\newpage
\section{List of Equipment}
To get an experimental compare to the theoretical equations of motion the following Equipment is used:
\begin{itemize}
\item The swing device is an torsional harmonic oscillator which has a flywheel at the end of the spring.
\item The Angular of the wheel is measured by an Angular to Voltage Converter which is connected to an PC-Oscilloscope. The coupling of the two wheels is realized by just using a small rope between the wheels.
\item The evaluation of the detected signals is made with the CASSY-Lab-software on the spot. 
\end{itemize}
\newpage
\section{Exercise 1}
In this task the oscillator will be deflected by hand to an amplitude near the maximum. The position resp. Angular of the flywheel Is measured and plotted with CASSY-Lab. 
\subsection{Observation}
The measured curves fit nearly perfectly the expectations of a exponential decreasing sin function. To process the data The envelope function will is fitted with a \(f(x)=A*e^{-x/B}+C\) function. The frequency  is measured just by counting the periods per time interval.
\subsection{Errors}
The equipment does not provide data which is totally free of systematic errors. There is a lot of friction which is not only dependent on the wheels velocity. But the systematic errors are not so significant that the data can not be evaluated.

It is assumed that the time is measured relatively correct \(\Delta t = 0.01\, s\). 
\subsection{Measured Data}
\begin{center}
\begin{tabular}{c|cccccc}
Position & \(B [s]\) & \(N\) & \(t_1 [s]\) & \(t_2 [s]\) & \(T [s]\) & \(\omega_1 [\frac{1}{s}]\) \\ \hline
3 & \(5.12\) & \(6\) & \(1.72 \pm 0.01\) & \(13.6 \pm 0.01\) & \(11.88\) & \(0.50505 \pm 0.00060\) \\ 
2 & \(5.41\) & \(6\) & \(13.90 \pm 0.01\) & \(1.68 \pm 0.01\) & \(12.22\) & \(0.49099 \pm 0.00056\) \\ 
1 & \(27.13\) & \(20\) & \(42.00 \pm 0.01\) & \(4.22 \pm 0.01\) & \(37.78\) & \(0.52938 \pm 0.00019\) \\ 
0 & \(69.13\) & \(20\) & \(50.82 \pm 0.01\) & \(13.16 \pm 0.01\) & \(37.66\) & \(0.53106 \pm 0.00019\) \\
\end{tabular}
\captionof{table}{measured values for free damped oscillations}
\end{center}
\subsection{Evaluation}
To determine the damping constant \(\delta\) the formula \eqref{x(t)_1} can be compared with the fitting function \(f(x)=A*e^{-x/B}+C\) and comparison gives:
\begin{align}
\delta = \frac{1}{B}
\end{align}
Moreover the eigenfrequency \(\omega_0\) can be determined with \eqref{w1} as:
\begin{align}
\omega_0 = \sqrt{\omega_1^2 + \delta^2}
\end{align}
\newpage
\section{Aufgabe 2}
In dieser Aufgabe soll die Schallgeschwindigkeit und der Isentropenindex bzw. Adiabatenkoeffizient über die Resonanzfrequnzen in einem Rohr bestimmt werden.
\subsection{Aufbau}
Vorhanden ist ein Rohr, dass auf der einen Seite einen Lautsprecher und auf der anderen Seite eine Öffnung hat, die man jedoch mit einem Brettchen verschließen kann. Der Lautsprecher wird mit Bananensteckerkabeln mit dem Funktionsgenerator verbunden um bestimmte Schallwellen im Rohr zu erzeugen. Gemessen wird qualitativ die Lautstärke an einer kleinen Öffnung am Rohr indem ein Mikrophon an ein Wechselspannungsmessgerät angeschlossen wird.
\subsection{Durchführung}

\subsection{Fehlerabschätzung}
\subsection{Gegebenes}
\subsection{Messwerte und Graphen}
\subsection{Auswertung}
\subsection{Fazit und Vergleich}

\newpage
\section{Conclusion}
The investigation of harmonic oscillations was very successful. The theory describes
the real motion of the system very well and both methods, the measuring of free oscillations
and driven oscillations gave the same results for damping constants and eigenfrequencies.


The qualitative investigation of the phase shift and the transient responses give
a good idea how complex even relatively simple differential equations represent. The
observed motion fits perfecly with the theory.


\end{document}