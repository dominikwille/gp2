\section{Aufgabe 1}
Bei dieser Aufgabe werden vor allem Einstellungen gesucht, bei denen eine Aufnahme der Frank-Hertz-Kurve in möglichst schöner Form möglich sind. Dabei hängt das Aussehen der Kurve von verschiedenen Faktoren ab. Verwendet werden ein Funktionsgeneartor, der 
\begin{itemize}
\item[] \textbf{Heizspannung:} Da die Heizspannung für ein Austreten von Elektronen aus der Kathode sorgt, gibt es einen Zusammenhang zwischen dem Strom von Anode zu Gegenkathode \(I\) und der Heizspannung \(U_H\). Es ist eine genügend große Heizspannung zu wählen, um ein Austreten von Elektronen zu ermöglichen, ist diese erreicht hat eine weitere Erhöhung nur noch wenig Effekt, sorgt im Allgemeinen aber für eine Erhöhung des Stroms \(I\)
\end{itemize}