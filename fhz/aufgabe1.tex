\section{Aufgabe 1}
Bei dieser Aufgabe werden vor allem Einstellungen gesucht, bei denen eine Aufnahme der Frank-Hertz-Kurve in möglichst schöner Form möglich sind. Dabei hängt das Aussehen der Kurve von verschiedenen Faktoren ab. Verwendet werden ein Funktionsgeneartor, der 
\begin{itemize}
\item[] \textbf{Heizspannung:} Da die Heizspannung für ein Austreten von Elektronen aus der Kathode sorgt, gibt es einen Zusammenhang zwischen dem Strom von Anode zu Gegenkathode \(I\) und der Heizspannung \(U_H\). Es ist eine genügend große Heizspannung zu wählen, um ein Austreten von Elektronen zu ermöglichen, ist diese erreicht hat eine weitere Erhöhung nur noch wenig Effekt, sorgt im Allgemeinen aber für eine Erhöhung des Stroms \(I\).
\item[] \textbf{Beschleunigungsspannung:} Die Beschleunigungsspannunng ist die Spannung \(U_B\), die zweischen Anode und Kathode angelegt wird. Sie sorgt dafür, dass sich die Elektronen in einem Elektrischen Feld befinden und somit \textit{beschleunigt} werden.

Bei diesem Versuchsaufbau wird die Beschleunigungsspannung variiert, indem eine Wechselspannung verwendet wird.
\item[] \textbf{Temperatur:} Die Temperatur wird bei dem Vorhandenen Versuchsaufbau eingestellt, indem ein Thermostat an der Apparatur selbst eingestellt wird, der eine elektrische Heizung steuert. Da der Thermostat sehr langsam reagiert ist die Temperatur nur schwer genau einstellbar und ändert sich selbst während einer Messung erheblich.

Sie sorgt dafür, dass das Quecksilber in den gasförmigen Zustand übergeht, da sonst kein ausreichender Abstand der Atome gegeben ist.
\end{itemize}
\subsection{Aufbau}
\subsection{Durchführung}
\subsection{Fehlerbetrachtung}
\subsection{Auswertung}
\subsection{Fazit}