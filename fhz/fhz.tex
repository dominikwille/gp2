\documentclass{article}
\usepackage{indentfirst}
\usepackage{lmodern}
\usepackage[utf8]{inputenc}
\usepackage[T1]{fontenc}
\usepackage[ngerman]{babel}
\usepackage{amssymb,amstext,amsmath}
\usepackage{graphicx}
%\usepackage{dsfont}
%\usepackage{amsfonts}
\usepackage{graphics}
\usepackage{float}
\usepackage{cite}
\usepackage{url}
\usepackage{tabularx}
\usepackage{capt-of}
%\usepackage{hyperref}
\usepackage{hyperref}

\title{Franck-Herz-Versuch}
\author{Alexander Heinisch, Dominik Wille}
\begin{document}
\maketitle

\begin{center}
\begin{minipage}{\linewidth}
\centering
\makebox[0cm]{\includegraphics[width=7cm]{bilder/fhz0}}
\label{wtd}
\end{minipage}
\end{center}

\vspace{7cm}
\noindent
\begin{center}
\begin{tabular}{r l}
Tutor & Theodore Haimberger\\
Durchführung & 05. Juni 2013 von 14-18 Uhr \\

E-Mail Dominik & dominik.wille@fu-berlin.de \\
E-Mail Alexander & Matthias.Heinisch@gmx.de \\
\end{tabular}
\end{center}

\newpage
\tableofcontents
\newpage

\section{Physikalische Grundlagen}
\subsection{Atommodell}
Am Anfang des 20. Jahrhunderts entdeckten die Physiker James Franck und Gustav Hertz diskrete Anregungsstufen  von Elektronen in Quecksilber. Diese Erkenntnisse waren ein glänzender Beweis für die neue Quantenhypothese und des Bohr-Sommerfeld-Atommodells.\\
In diesem Atommodell nahm Bohr an, dass sich die Elektronen nur auf bestimmten Bahnen strahlungsfrei bewegen können. Dabei entsprechen die Bahnen einem vielfachen der DeBroglie-Wellenlänge (Materiewelle der Elektronen).\\
Demnach können ein Großteil der rechnerischen Bahnen ausgeschlossen werden und es sind nur diskrete Werte für den Drehimpuls und den Bahnradius möglich. Später fand Sommerfeld heraus, dass sich die Elektronen auch auf Ellipsenbahnen bewegen können und nicht nur, wie zuvor angenommen, auf Kreisbahnen. Damit die Bahnen eindeutig parametrisiert werden können, ordnete man ihnen die Quantenzahlen n,l,m,s zu. Dabei beschreibt die Hauptquantenzahl n die Energieniveaus der Elektronen mit den Werten n=1,2,3,... Je größer die Zahl wird, desto weiter ist der Abstand zu dem Kern des Atoms. l ist die Nebenquantenzahl, welche die Exzentrizität der Ellipse im Wertebereich \(o\leq l\leq n-1\). Die Magnetquantenzahl m beschreibt die Ausrichtung der Ellipsen, wodurch die Ellipse nun dreidimensional beschrieben ist. Die Spinquantenzahl s nimmt Werte von \(\pm\frac{1}{2}\) an und beschreibt den Eigendrehimpuls des Elektrons. Durch das Pauli-Prinzip kann nur ein Elektron jeweils ein Energieniveau besetzten.

\subsection{Termschema}
Die Elektronen befinden sich bekanntlich auf verschiedenen Energieniveaus, welche von Innen (niedrige Energie) nach Außen (hohe Energie) besetzt werden. Diese Elektronen können durch Anregung auf ein höheres Energieniveau gehoben werden. Wenn die angeregten Elektronen wieder auf ihren Grundzustand zurückfallen, emittieren sie Photonen, deren Wellenlänge der Energiedifferenz der Niveaus entspricht. Die folgende Abbildung veranschaulicht, über welche Niveaus ein Elektron auf seinen Grundzustand zurückfallen kann:

\begin{center}
\begin{minipage}{\linewidth}
\centering
\makebox[0cm]{\includegraphics[width=7cm]{bilder/fhz2}}
\captionof{figure}{Allgemeines Termschema}%
\label{term}
\end{minipage}
\end{center}

\subsection{Frank-Hertz-Versuch}
\begin{center}
\begin{minipage}{\linewidth}
\centering
\makebox[0cm]{\includegraphics[width=7cm]{bilder/fhz1}}
\captionof{figure}{Versuchsaufbau}%
\label{skizze}
\end{minipage}
\end{center}
Bei diesem Experiment hatten sie eine mit Quecksilberdampf gefüllte Röhre, in welcher Elektronen von einer Glühkathode K aus beschleunigt werden und durch ein Gitter G zu einer Auffangelektrode A gelangen. Die Beschleunigung erfolg mithilfe einer Beschleunigungsspannung \(U_B\). Sie ist bedeutend größer als die zwischen dem Gitter und der Anode anliegenden Bremsspannung \(U_G\).


Anfangs erfolgen nur elastische Stöße der Elektronen mit dem Quecksilber und im Auffängerkreis ist eine gleichmäßig zunehmender Stromfluss messbar. Erreicht die Beschleunigungsspannung jedoch einen Schwellenwert bzw. ihr ganzzahliges Vielfaches, so bricht der Strom ab und steigt anschließend wieder von neuem an (siehe Abbildung 3).


Diese Abbrüche werden dadurch verursacht, dass die Energie der Elektronen an diesen Stellen der Energiedifferenz eines Übergangs zum nächsten Niveau entspricht. Steigt die Energie weiter, findet ein inelastischer Stoß zwischen Elektronen und Gasmolekülen statt, wobei die kinetische Energie vom Elektron auf das Gasmolekül übertragen wird. Das hat zur Folge, dass der Stromfluss zum erliegen kommt, was eben durch die Minima im Diagramm veranschaulicht wird. Die Freiwerdende energie wird als Elektromagnetische Welle abgegeben.


Im Praktikum werden wir Bariumoxidkathoden benutzen und die in der Röhre angeordneten Elektroden sind planparallel angeordnet. Der Abstand zwischen Kathode und Anode ist größer als die mittlere freie Wellenlänge, wodurch die Stoßwahrscheinlichkeit erhöht wird.

{\begin{center}
\begin{minipage}{\linewidth}
\centering
\makebox[0cm]{\includegraphics[width=7cm]{bilder/fhz3}}
\captionof{figure}{Theoretischer Kurvenverlauf}%
\label{verlauf}
\end{minipage}
\end{center}


Der Literaturwert der Spannung für Quecksilber beträgt \(\Delta\)U = 4,9eV.\\
Für Neon hingegen liegt der Wert bei \(\Delta\)U = 18,9eV.

\newpage
\section{Aufgaben}
\subsection*{Aufgabe 1}
Beobachtung der Elektronenstoß-Anregungskurve (Franck-Hertz-Kurve) von Quecksilber bei einer (Ofen-)Temperatur von etwa 190\(^\circ\)C mit dem Oszilloskop. Optimierung der Kurve durch geeignete Einstellungen der experimentellen Parameter (Ofenheizung, Kathodenheizung, Beschleunigungsspannung)

\subsection*{Aufgabe 2}
Quantitative Aufnahme der Kurve mit einem X-Y-Schreiber. Bestimmung der zugehörigen Übergangsenergie in Quecksilber.

\subsection*{Aufgabe 3}
Beobachtung und Registrierung weiterer Anregungskurven für Temperaturen von 150 und 210\(^\circ\)C. Qualitative Diskussion der Ergebnisse.

\subsection*{Aufgabe 4}
Aufnahme und Auswertung einer Franck-Hertz-Kurve für Neon bei Zimmertemperatur.

\newpage
\section{Aufgabe 1}
Bei dieser Aufgabe werden vor allem Einstellungen gesucht, bei denen eine Aufnahme der Franck-Hertz-Kurve in möglichst schöner Form möglich sind. Dabei hängt das Aussehen der kurve von verschiedenen Faktoren ab. Verwendet werden ein Funktionsgeneartor, der 
\begin{itemize}
\item[] \textbf{Heizspannung:} Da die Heizspannung für ein Austreten von Elektronen aus der Kathode sorgt, gibt es einen Zusammenhang zwischen Dem Strom von Anode zu Gegenkathode \(I\) und der Heizspannung \(U_H\). Es ist eine genügend große Heizspannung zu wählen um ein Austreten von Elektronen zu ermöglichen, ist diese erreicht hat eine Weitere Erhöhung nur noch wenig Effekt sorgt im Allgemeinen aber für eine Erhöhung des Stroms \(I\)
\end{itemize}
%\section{Aufgabe 2}

Aufgrund veränderter Umstände, haben wir die Frank-Hertz-Kurven nicht mehr mit einem X-Y-Schreiber aufgezeichnet, sondern mit einem Programm auf den Rechnern im Versuchsraum.
%\subsection{Aufgabe 3}
\begin{figure}[H]
\includegraphics[scale=0.5]{sb3}
\caption{Abbildung 1: Schaltplan Aufgabe 3}
\end{figure}
In der letzten Aufgabe wurde eine Wechselstrombrücke aufgebaut um nun die Induktivität L einer Spule zu bestimmen. Der Aufbau besteht aus einem Phasenabgleichswiderstand R' mit einer bekannten und einer unbekannten Spule \(L_{0}\) und \(L_{x}\) in Reihe geschalten und dazu parallel ist die Wechselstrombrücke. Das Messgerät wird zwischen R' und der Wechselstrombrücke platziert. Das Ziel ist nun eigentlich, den Strom durch die Wechselstrombrücke auf Null zu stellen, was allerdings wegen den hier verwendeten Bauteilen nicht ganz möglich ist. Also versucht man ihn zu minimieren. Gemessen hatten wir mit dem Messgerät ELC-131D (\(\pm 2\% +5\) dgt) für \(\Delta L_{VS}\) und (\(\pm 0,5\% +3\) dgt) für \(\Delta R_{VS}\)\\
Bei einer Frequenz von f=1.9989 kHz erhielten wir für die gesuchten Größen folgende Werte:
\\

\begin{tabular}{l l}
\(R_{a}\) & =\(761,0\pm 4,3 \Omega\)\\
\(R_{b}\) & =\(241,8\pm 1,7 \Omega\)\\
\(R'\) & =\(3,700\pm 0,023 \Omega\)\\
\end{tabular}
\\

Für die Vergleichsspule gelten die Werte:\\

\begin{tabular}{l l}
\(R_{VS}\) =\((2,91\pm 0,02)\Omega\)\\
\(L_{VS}\) =\((1,507\pm 0,015)\)mH\\
\end{tabular}
\\

Nun überprüftt man mit dem Oszilloskop, ob die Spulen mit dem Funktionsgenerator in Phase sind.\\
\newpage
Des weiteren gilt es nun die Induktivität L und den Verlustwiderstand \(R_{V}\) der unbekannten Spule berechnen. Dazu nimmt man Gl.\(\eqref{L}\) für \(L_{x}\):
\begin{equation}\notag
L_x = {\frac{R_a}{R_b}} L_0=(4,75\pm 0,07)mH
\end{equation}

und auf Grund der Phasengleichheit errechnet sich für \(R_{V}\):
\begin{equation}\notag
R_{V}=\frac{R_{a}}{R_{b}}\cdot R_{VS}=(9,16\pm 0,11)\Omega
\end{equation}
Für die Fehler gilt:
\begin{equation}\notag
\Delta L_x = {\sqrt{\left({\frac{\partial L_x}{\partial R_a}}\Delta R_a\right)^2+\left({\frac{\partial L_x}{\partial R_b}}\Delta R_b\right)^2+\left({\frac{\partial L_x}{\partial L_0}}\Delta L_0\right)^2}}
\end{equation}
\begin{equation}\notag
\Delta R_{L_x} = {\sqrt{\left({\frac{\partial R_{L_x}}{\partial R_a}}\Delta R_a\right)^2+\left({\frac{\partial R_{L_x}}{\partial R_b}}\Delta R_b\right)^2+\left({\frac{\partial R_{L_x}}{\partial R_{L_0}}}\Delta R_{L_0}\right)^2}}
\end{equation}
\\

Als letztes wird noch der theoretische Werte für den Phasenunterschied bestimmt. Gemessen hatten wir für die Induktivität L:
\begin{equation}\notag
L=(4,823 \pm 0,038)mH
\end{equation}
\begin{equation}
\notag
R_{L}=(162,5 \pm 1,3)\Omega
\end{equation}

Dies nun in Gl.\(\eqref{d}\) eingesetzt mit \(\omega=2\pi f\) (f=199,98 Hz) und nach \(\phi\) aufgelöst, ergibt:
\begin{equation}
\notag
\phi=arctan \left(\frac{2\pi f L}{R_{L}}\right)=(2,14 \pm 0,03)^\circ
\end{equation}

Für den Fehler gilt wieder:
\begin{equation}\notag
\Delta \phi = {\sqrt{({\frac{\partial \phi}{\partial R_{L_1}}}\Delta R_{L_1})^2+({\frac{\partial \phi}{\partial L_1}}\Delta L_1)^2}}
\end{equation}

\subsection*{Fazit}
An sich verlief bei dieser Aufgabe alles soweit ganz gut, bis auf das kleine Zeitproblem was alle Gruppen bei diesem Versuch hatten. Da wir auch die einzige Gruppe waren, die den Aufbau zu dieser Aufgabe geschafft hatten, durften wir sozusagen den Versuch vorführen. Das ging dann allerdings ein wenig chaotisch zu, weil immer nachgefragt wurde was jetzt nochmal welcher Wert genau gewesen war. Nichts desto trotz sehen unsere Ergebnisse für Diesen Versuch ganz ordentlich aus. Das einzige was komisch ist, ist dass unser Widerstand der Vergleichsspule so klein ist. Da uns aber Vergleichswerte fehlen, beruht diese Aussage eher auf ein Bauchgefühl, als auf Tatsachen.\\
In der Berechnung des Fehlers der Phasenverschiebung haben wir den Fehler für die Frequenz weggelassen, weil er im Verhältnis zu denen der anderen Werte so verschwindend gering ist, dass es nicht ins Gewicht gefallen wäre.\\
Alles in allem verlief dieser Versuch zufriedenstellend.
\subsection{Aufgabe 4}

In unserem Versuch der Photoemission, werden einige Widersprüche zur klassischen Wellentheorie deutlich und unsere erhaltenen Ergebnisse stimmen mit Einsteins Quantenhypothese überein.\\

Die klassische Wellentheorie besagt zum Beispiel, dass es keinen Zusammenhang zwischen der Intensität des Photostroms und der Anzahl der herausgelösten Elektronen des Metalls gibt.\\
Dies wurde durch die in Aufgabe 2 gemessenen Werte eindeutig widerlegt. Je höher die Intensität des Photostroms, desto höher ist auch das Maß der emittierten Elektronen.\\
Auch besagt die Wellentheorie, dass die kinetische Energie der Photonen direkt von der Intensität des einfallenden Lichts abhängig ist. Wenn ein Elektron von einer oszillierenden Schwingung angeregt werden würde, müsste es das Metall verlassen, wenn eine kritische Amplitude überschritten wäre. Das Elektron hätte also demnach eine gewisse kinetische Energie. Des weiteren hat diese Schwingung die Folge, dass es erst kurze Zeit nach dem Einschalten einen Messbaren Elektronenstrom gibt. Unseren Beobachtungen zufolge, war nach dem Einschalten instantan ein konstanter Elektronenstrom vorhanden.\\
Die Erkenntnisse der Photoemission belegen jedoch, dass in Wirklichkeit die Frequenz des Lichts einen direkten Einfluss auf die kinetische Energie der Elektronen hat.\\
Auch die in Aufgabe 3 erhaltenen Messwerte liefern einen eindeutigen Zusammenhang zwischen Frequenz und Bremsspannung, was der klassischen Ansicht nach unmöglich ist.

\section{Fazit}
Alles in allem verlief der Franck-Hertz-Versuch sehr zufriedenstellend. Die Messungen beim Quecksilber lieferten zwar signifikant verschiedene Werte zum Literaturwert, das liegt aber höchstwahrscheinlich nur an der zu geringen Wartezeit, damit die Gesamte Apparatur eine Temperatur annimmt und an einem zu klein abgeschätzten Fehler. 

Das prinzipielle physikalische Phänomen, der Diskreten Anregungsenergien der Atome kann absolut bestätigt werden, die aufgenommenen Franck-Hertz-Kurven entsprechen den Erwartungen in vollem Umfang.

\section{Qellenangabe}
\begin{itemize}
\item GPII-Skript
\item Platzskript
\item \(http://grundpraktikum.physik.uni-saarland.de/scripts/New_features.pdf\)
\end{itemize}
\vspace{7.0cm}

\begin{tabularx}{\textwidth}[b]{p{5cm} X p{5cm}} \cline{1-1} \cline{3-3}
Datum, Dominik Wille & & Datum, Alexander Heinisch
\end{tabularx}

\end{document}
